\chapter{Entwurf der Benutzerschnittstelle}\label{chapter_4}
Die Anforderungen der Anwendung für die Unterstützung des neuen Workflows wurden im vorigen Kapitel abgeleitet. Diese Kriterien werden im nächsten Schritt in einem Entwurf der Benutzerschnittstelle umgesetzt. Bevor die Schnittstelle entworfen werden kann, muss die geeignete Anwendungsform und eine darauf basierende Technologie ausgewählt werden. Anschließend folgt eine kurze Analyse der Zielplattform und deren Konzepte, damit diese beim Entwurf berücksichtigt werden können, bevor die einzelnen Ansichten entworfen werden.

\section{Auswahl der Anwendungsplattform}
Aufgrund der vielen Möglichkeiten bei der Entwicklung von mobilen Anwendungen muss die Art und Technologie der Anwendung richtig gewählt werden. Hierbei soll zuerst die passende Anwendungsform (Nativ, Web oder Hybrid siehe \ref{mobileAppsGrundlagen}) gewählt werden. Durch diese Vorauswahl, wird die Anzahl der möglichen Technologien begrenzt, was im folgenden Schritt die Auswahl vereinfacht.


\subsection{Anwendungsform}
Damit die richtige Anwendungsform der App ausgewählt werden kann, müssen die drei Möglichkeiten Nativ, Web und Hybrid auf ihre Eignung bei der Umsetzung der Anforderungen untersucht werden. Wichtig bei der Entscheidung ist die Festlegung der konkreten Kriterien. 
Die Umsetzung der rein fachlichen Funktionen Produktauswahl und Konfiguration kann mit jeder Anwendungsform durchgeführt werden. Hier sind die Unterschiede für eine Auswahl nicht ausreichend. \par
Bei den  Funktionalen Anforderungen sind die zwei Kriterien, die aufgrund der mobilen Zielumgebung wichtig sind für die Auswahl bedeutend. Diese sind das Speichern und Laden (F5), sowie die Filterung der Anwendungsdaten (F6). Für die Umsetzung dieser Funktionen ist eine Verwendung des Dateisystems auf der mobilen Zielumgebung notwendig. Hier muss eine Form der Speicherung, ob Datenbank oder einfaches Speichern in einer Datei möglich sein. Diese Voraussetzung ist für die Entscheidung der Anwendungsform essentiell. \par

Das zweite Kriterium leitet sich von den Nicht-Funktionalen Anforderungen einer schnellen Bedienung (N2) ab. Damit die Anwendung schnell bedient werden kann, müssen die vorhandenen Hardwareressourcen optimal verwendet werden können. Dies ist notwendig, um bspw. das Laden von Bildern zu beschleunigen. Ebenfalls müssen Seitenübergänge ohne große Wartezeiten möglich sein, um die Anforderung erfüllen zu können. Die Anwendungsform muss für die Umsetzung Schnittstellen bereitstellen, die auf die vorhandene Hardware optimiert sind.

Für eine Optimierung der Anwendung auf die Zielumgebung (N3) ist ein ästhetisches und minimalistisches Design die entscheidende Heuristik für die Auswahl der Anwendungsform. Für die Implementierung muss eine gute optische Integration in das System vorhanden sein. Dies ist für eine übersichtliche Aufbereitung des Produktkataloges eine Voraussetzung. Der App-Typ muss Oberflächenelemente zur Verfügung stellen, die zu der Gesamtumgebung passen. Durch eine gute Integration in die Zielumgebung wird dadurch die Anwendung ästhetischer. Die gesamte Wahrnehmung der App wird hierdurch verbessert. Das dritte Zielkriterium für die Auswahl ist somit die Verwendung von betriebssystemspezifischen Oberflächenelementen.    

Im Folgenden werden die drei festgestellten Kriterien lokale Speicherung der Daten, Hardwarenahe Schnittstellen und Verwendung von betriebssystemspezifischen Oberflächenelementen  für jede Anwendungsform bewertet, sodass am Ende eine Gegenüberstellung stattfinden kann.

\paragraph{Native Anwendung: }Bei Nativen Anwendungen ist der komplette Funktionsumfang der mobilen Zielplattform verfügbar. Dies ermöglicht ein breites Anwendungsfeld für native Anwendungen.
\begin{itemize}
\item \textbf{Lokale Speicherung der Daten:} Die native Anwendungsform stellt Schnittstellen für den Zugriff auf eine lokale Datenbank oder ein lokales Dateisystem bereit. Diese können ohne Anpassungen verwendet werden. 

\item \textbf{Hardwarenahe Schnittstellen:} Dadurch, dass die einzelnen Schnittstellen direkt vom Hersteller der jeweiligen Plattform kommen, sind Operationen für das System optimiert. Mit einer nativen Anwendung wird damit die beste Performanz erreicht. 

\item \textbf{Betriebssystemspezfische Oberflächenelemente:} Native Anwendungen verwenden für die Benutzerschnittstelle die spezifische Oberflächenelemente. Die Steuerung mit Touch ist ebenfalls auf diese Anwendungsform optimiert. Dies ergibt eine vollständige Verwendungsmöglichkeit des bereitgestellten Frameworks.
\end{itemize}

Auf die einzelnen Kriterien bezogen erfüllt die native Anwendung alle Anforderungen. Der volle Funktionsumfang des Systems ist gegeben, wodurch die App ideal für die gewählte Zielplattform geeignet ist.

\paragraph{Web Anwendung: } Web Anwendungen werden im Browser ausgeführt. 
 Dies führt dazu, dass nur Funktionen verwendet werden können, die der jeweilige Browser auf der Zielplattform zur Verfügung stellt. 
 \begin{itemize}
 \item \textbf{Lokale Speicherung der Daten:} Ein Speichern und Laden der Daten vom Betriebssystem ist bei einer Web Anwendung nicht ohne weiteres möglich. Der Browser verbietet durch das Sandbox-Modell einen direkten Zugriff auf das System.  Ein Ansatz zur Lösung des Problems ist die Installation des Servers direkt auf dem Zielsystem, so dass dieser lokal zur Verfügung steht. Eine Erfüllung der Anforderungen ist damit möglich, jedoch mit erheblichem Mehraufwand. 
 
 \item \textbf{Hardwarenahe Schnittstellen:} Die Kommunikation mit dem Betriebssystem erfolgt durch den Browser. Diese zusätzliche Zwischenschicht sorgt dafür, dass die Hardware nicht direkt verwendet wird, wie es bei einer nativen Anwendung der Fall ist. Der zusätzliche Overhead sorgt dafür, dass die Performance bei einer Web Anwendung nicht ideal im Vergleich zur nativen Lösung ist.
 
 \item \textbf{Betriebssystemspezfische Oberflächenelemente:} Die grundlegenden Oberflächenelemente sind HTML Elemente. Diese können mit Javascript und CSS einen passenden Stil für das jeweilige Betriebssystem erhalten. Die Verwendung von speziellen Touch Gesten ist mit Web Anwendungen aufgrund der Interoperabilität mit mehreren Betriebssystemen schwieriger. Hier sind ebenfalls nur begrenzte Interaktionen möglich.
 
 \end{itemize}
 Der Hauptvorteil der Web Anwendung die Verwendung, ohne komplette neue Implementierung auf mehreren Systemen ist aufgrund der Kriterien nicht relevant. Die wichtigen Eigenschaften der Anwendung Performanz und lokale Speicherung können mit dieser Anwendungsform gelöst werden, jedoch deutlich schlechter als bei der nativen Variante.


\paragraph{Hybride Anwendung: }Durch die Verwendung der lokalen Schnittstellen im nativen Anwendungscontainer (siehe \ref{hybridApplication}) können die vorhanden Ressourcen wie bei einer Nativen Anwendung verwendet werden. Dies führt zu einigen Verbesserungen gegenüber einer reinen Web Anwendung.

\begin{itemize}
 \item \textbf{Lokale Speicherung der Daten:} Für die Umsetzung dieser Anforderung muss eine geeignete Schnittstelle im Anwendungscontainer zur Verfügung gestellt werden. Hierdurch wird eine problemlose Verwendung des Dateisystems möglich. Das Kriterium kann ohne Einschränkungen erfüllt werden. 
 
 \item \textbf{Hardwarenahe Schnittstellen:} Durch den Anwendungscontainer wird, wie bei einer Webanwendung der Browser, eine Zwischenschicht nötig. Damit wird ein zusätzlicher Kommunikationsaufwand benötigt. Dieser zusätzliche Aufwand verringert die Performanz der Anwendung.
 
 \item \textbf{Betriebssystemspezfische Oberflächenelemente:} Die Oberflächenelemente des Betriebssystems lassen sich im Container bereitstellen. Somit können die nativen Elemente in der Hybriden Anwendung verwendet werden. Die Verwendung der Touch Bedienung wird durch dieses Konzept ermöglicht.
 \end{itemize}
 Die Probleme, die bei der Verwendung einer Web Anwendung auftreten können durch den hybriden Ansatz gelöst werden. Ein Problem was weiterhin besteht, ist die Performanz. Diese wird mit der hybriden Anwendungsform besser, kommt jedoch nicht an die Leistung einer nativen Anwendung heran.

\paragraph{Abwägung: }
Nachdem alle Anwendungsformen auf die Erfüllung der aufgestellten Kriterien untersucht sind, kann eine Gegenüberstellung der einzelnen Komponenten erfolgen.  Bei der ersten Betrachtung der Anwendungsformen ist die Implementierung als Web Anwendung nicht sinnvoll. Die Anforderungen können zwar umgesetzt werden, jedoch ist ein erheblicher Mehraufwand für die Erreichung nötig. Ebenfalls hat die Anwendung im Vorhinein bereits Einschränkungen bzgl. der Performanz, sowie der Bereitstellung von Oberflächenelementen. Die Vorteile einer Implementierung für mehrere Zielsysteme ist bei der Umsetzung kein Kriterium. \par

Schwieriger ist die Entscheidung zwischen hybrid und nativ. Die beiden Kriterien lokale Speicherung der Daten und betriebssystemspezifsche Oberflächenelemente sind mit beiden Ansätzen ohne größere Einschränkungen möglich. Beim hybriden Ansatz ist ein geringer Mehraufwand bei der Implementierung nötig. Dies ist jedoch nicht relevant für die Entscheidung. Der wichtigste Grund für die Entscheidung für eine nativen Anwendung ist die Performanz bei der Bedienung. Nur durch die Verwendung von nativen Anwendungen können aufwändige Animationen oder das Rendern von vielen Bildern flüssig ablaufen. Eine hybride Lösung besitzt hier Einschränkungen, da die Hardware mit einer Zwischenschicht verwendet wird.

\subsection{Anwendungstechnologie}
Mit der Entscheidung für eine mobile Anwendungsform kommen drei mögliche Technologien für die Implementierung in Frage. Die Plattformen Android \footnote{http://www.android.com/}, iOS \footnote{http://www.apple.com/de/iphone/ios/} und Windows 8 \footnote{http://windows.microsoft.com/de-de/windows-8/} sind aufgrund ihrer Marktanteile (Android: 43,4\% iOS: 48,2\% Windows: 7,4\% Quelle: \cite{bib:marktanteilBS} ) die wichtigsten Technologien im Tablet Bereich. Die drei Plattformen haben unterschiedliche Ziele. Das iOS Betriebssystem hat den Vorteil einer hohen Verbreitung bei Business Anwendungen (siehe \cite[S.5]{bib:mobileMarketing2}). Android hat die meisten Nutzer bei Privatanwendungen (siehe \cite{bib:marktanteilMBS}). Bei Windows 8 ist der Vorteil eines neuen Konzeptes, was speziell für Tablet-PCs optimiert ist. Für die Umsetzung des Workflows kann jede dieser Technologien verwendet werden. Ist die Implementierung mit einem Framework umgesetzt, stellt das Implementieren auf einer anderen Zielplattform keine Herausforderung da. \par 

Bei der vorliegenden Arbeit wird auf Windows 8 gesetzt. Im Gegensatz zu iOS und Android wird dieses Betriebssystem nicht auf Smartphones ausgeführt. Dies führt zu einer besseren Optimierung für größere Bildschirme. Die Technologie ist sehr neu auf dem Markt (26. Oktober 2012), dadurch gibt es noch nicht viele Apps. Dies bietet die Möglichkeit den Anwender mit neuen Möglichkeiten in der Anwendung zu überraschen, da einige Konzepte noch nicht bekannt sind. Die Möglichkeit zu experimentieren und neue Ideen umzusetzen ist aufgrund der neuen Plattform vorhanden.  Die Verwendung von Windows 8 vereinfacht ebenfalls die Integration in ein Unternehmensumfeld, da hier die Microsoft Produkte weit verbreitet sind. Dies führt zu einer schnelleren Akzeptanz im Unternehmen. Somit wird im Folgenden die Anwendung für das Windows 8 Betriebssystem implementiert.
        

\section{Untersuchung der Plattform Windows 8}
Damit die Anforderung eines ästhetischen und minimalistischen Designs (N3) erfüllt ist, muss vor der Gestaltung der einzelnen Ansichten die Zielplattform untersucht werden. Die Designgrundlagen müssen verstanden werden, um ein passendes Aussehen realisieren zu können. 
Die Folgende Untersuchung wird in drei Teile aufgeteilt. Im ersten Abschnitt werden allgemeine Design Prinzipien behandelt, darauf aufbauend die Bedienkonzepte und touchoptimierten Bedienelemente der Plattform.

\subsection{Design Richtlinien}
Damit einheitliche Apps entwickelt werden, hat Microsoft Richtlinien (Entnommen aus: \cite{bib:win80}, \cite{bib:win81}, \cite{bib:win82}, \cite{bib:win83}) aufgestellt.
Das wichtigste Design Element bei einer Windows 8 Anwendung sind sogenannte Kacheln. Diese Kacheln sind meist quadratisch und in jeder App vorhanden . Jede Funktion wird über eine Kachel erreicht. Sie soll mehr Informationen darstellen, als ein einfaches Logo oder Icon auf einem Button. Durch einen dynamischen Inhalt und unterschiedliche Größen wird dadurch dem Benutzer ein neues Benutzererlebnis gegeben. Die Organisation der Kacheln erfolgt in einem sogenannten Grid (engl. für Gitter). Dieses besteht aus mehreren Quadraten. Das kleinste Quadrat hat eine Größe von einem Pixel. Das Grid besitzt verschiedene Bereiche für die Überschrift und den Inhalt. Dieser ist vorgegeben und sollte eingehalten werden. \par 

Die Anwendung muss sich auf die Anzeige des Wesentlichen konzentrieren. Microsoft nennt  das Prinzip "'Content over Chrome"'.  Für die Umsetzung dieser Richtlinie soll die Anwendung nur die wichtigsten Funktionen in der Ansicht darstellen. Es sollen überladene Ansichten vermieden werden und stattdessen bewusst größere Elemente mit mehr Platz verwendet werden. 
 
Ein weiteres wichtiges Design Element ist die Typographie. Hier geht es um die bewusste Gestaltung von Schriften. Die Kalligrafie wird als Vorbild verwendet.  Die Idee ist keine rein statische Verwendung der Texte. Der Nutzer soll die Möglichkeit haben diese auszuwählen, damit eine Interaktion ermöglicht wird. 

Diese Richtlinien werden beim Entwurf der einzelnen Ansichten berücksichtigt. 


\subsection{Bedienkonzepte}
Bei den Konzepten für die Bedienung steht bei Windows 8 eine besondere Optimierung für Tablet-PCs im Vordergrund. Es werden deshalb sehr viele Gesten verwendet. Eine wichtige Geste ist das sogenannte "'wischen"'. Diese Aktion ist von jeder Seite des Bildschirms erlaubt. Beim Wischen von oben oder unten wird die sogenannte AppBar eingeblendet. Diese Bar wird an der oberen Seite für die Navigation durch die Anwendung verwendet. Dies ermöglicht einen schnellen Wechsel zwischen den Ansichten. Die untere AppBar wird für Aktionen verwendet, die nicht Vordergrund stehen. Ein Beispiel wäre die Filterung der Eingabedaten nach bestimmten Kriterien. Ein Wischen von der rechten Seit lässt die sogenannte CharmBar erscheinen. Der Inhalt dieser Bar ist die Verwendung von sogenannten Contracts (engl. Verträge). Diese Funktionen werden für alle Anwendungen durch das Betriebssystem bereitgestellt. An dieser Stelle können Einstellungen oder Suchen durchgeführt werden. \par 

Das zweite wichtige Bedienkonzept ist das horizontale Scrollen. Aufgrund der größeren Bedienelemente sind nicht immer alle Elemente sichtbar. Die Lösung ist ein horizontales Ausbreiten des Inhalts. Damit die Inhalte verwendet werden können, wird ein horizontales Scrollen mit einer Wisch-Geste durchgeführt. Hier muss darauf geachtet werden, dass ein Ausschnitt des nächsten Elementes sichtbar ist, damit dem Benutzer eine Erweiterung der Ansicht signalisiert wird. \par 

Beim Aufbau der Anwendung kann entweder eine hierarchische oder  flache Struktur verwendet werden. Für den in Abschnitt \ref{workflowNew} erstellen Workflow ist eine hierarchische Architektur passender. Der Ursprung geht hier immer von der Startseite, der sogenannten Hub-Page aus. Diese Seite ist der zentrale Startpunkt von der alle weiteren Aktionen ausgehen. Die zweite Ebene sind sogenannte Section Pages. Diese stellen den Inhalt einer  Kategorie dar. Die unterste Ebene sind die Detail Pages. Diese enthalten die jeweiligen Details eines Elementes in einer Kategorie. Bei einer Zeitungs App wäre beispielsweise auf der Startseite die einzelnen Kategorien wie Politik, Wirtschaft oder Sport zu sehen. In der Kategorie Ansicht die jeweiligen Überschriften der Artikel. Die Detail Seite würde den Artikel zeigen. Dieser baumartige Aufbau wird durch das Verwenden eines Zurück Buttons unterstützt, der in jeder Ansicht, außer der Startseite vorhanden ist. Durch diesen Button wird dem Benutzer eine weitere Navigationsmöglichkeit gegeben.

Beim Design der Ansichten ist ein durchdachtes Bedienkonzept aufgrund der beiden Anforderungen N1 und N2 wichtig. Die Möglichkeiten, die zur Verfügung gestellt werden, sollten beim ersten Entwurf der App enthalten sein.

\subsection{Touchoptimierte Bedienelemente}
Für die Unterstützung der Bedienung durch Gesten enthält das Framework besondere Oberflächenelemente. Diese sind für die Verwendung mittels Touch optimiert. Die in der Arbeit verwendeten Elemente werden im folgenden vorgestellt.
Die Design Richtlinien von Microsoft erzeugen Probleme bei der Darstellung von vielen Daten. Eine Umsetzung der Richtlinie "'Content over Chrome"', sowie die Darstellung auf einem Gerät mit kleinerem Bildschirm verursachen einen großen Aufwand beim Scrollen. Hier kann eine lange Zeit für das Auswählen eines bestimmten Datums benötigt werden. \par

Für die Lösung dieses Problems bietet Windows 8 den sogenannten Semantischen Zoom an. Diese Funktion wird mit einer Kneif-Geste auf dem aktuellen Datensatz durchgeführt. Hierdurch wird die Ansicht nicht optisch verkleinert. Es erfolgt ein Wechsel von der Detailansicht zu einer Kategorieansicht. Die Datensätze werden somit semantisch verkleinert, wodurch ein schnelles Navigieren zum gewünschten Datum möglich ist. Der Semantische Zoom darf nicht geschachtelt verwendet werden und ist damit auf eine Ebene beschränkt. Voraussetzung für die Verwendung des Zooms ist die Einteilung der Daten in Kategorien. Ohne diese Kategorien kann keine übergeordnete Ansicht erstellt werden.

Das zweite neue Bedienelement, welches für eine Touch-Optimierung dient, ist die sogenannte Flip-Ansicht. Mit ihr kann durch ein Wischen von der linken oder rechten Seite die Ansicht gewechselt werden. Dieses Element kann den Benutzer bei einem schnellen navigieren helfen. Es sollten jedoch nicht zu viele Elemente für den Wechsel vorhanden sein.

\section{Entwurf der Ansichten}
Die vorgestellten Konzepte der Anwendungsplattform, sowie die einzelnen Bedienelemente werden für den Entwurf der Ansichten verwendet. Der Workflow der App, wie in Abschnitt \ref{appWorkflow} beschrieben, wird hier als Grundlage verwendet. Alle Prozessschritte erhalten eine eigene Ansicht. Zusätzlich werden die Anforderungen aus Kapitel \ref{functionalRequirements} beim Entwurf berücksichtigt.
\subsection{Startseite}
\subsection{Produktkatalog}
\subsection{Flugzeugauswahl}
\subsection{Konfigurationsergebnisse}
\subsection{Navigationskonzept}
