\chapter{Einleitung}
\section{Motivation} \label{aufgaben}
Auf dem heutigen Käufermarkt wird der Verkauf von individuellen Produkten immer wichtiger. Da Produkte sich immer weniger in ihren Hauptfunktionen unterscheiden, muss dem Kunden die Möglichkeit gegeben werden sein Produkt individuell zusammenzustellen. Dieser Trend wird als \textit{Mass Customization} \cite{bib:massCustomization} bezeichnet. Die Konfigurationsmöglichkeit des Produktes verursacht zusätzlichen Aufwand, da während der Konfiguration die technische Realisierbarkeit geprüft werden muss. Produktkataloge, die Informationen über die Komponenten eines Produktes beinhalten, sind meist in Papierform vorhanden. Aus diesem Grund wird die Prüfung, ob das Produkt realisiert werden kann normalerweise von einem erfahrenen Mitarbeiter händisch durchgeführt.

\par Für die Unterstützung dieses Prozesses werden immer mehr computergestützte Systeme eingesetzt. Diese verwenden modellierte Regeln, die eine vorhandene Produktwelt abbilden. Für die Anzeige, bzw. das Zusammenstellen des Produktes werden vermehrt mobile Endgeräte verwendet. Durch ihre hohe Verfügbarkeit, sowie einfache Handhabung bieten sie entscheidende Vorteile beim Verkaufsgespräch mit dem Kunden. Ein Fokus bei der Entwicklung einer Anwendung für ein mobiles Endgerät (im folgenden App genannt) ist eine intuitive Bedienung durch Touch-Gesten.  Ein passendes Bedienkonzept hilft beim Durchführen einer Zusammenstellung für den Kunden (Konfiguration). Die Herausforderung besteht im Zusammenführen der komplexen Konfigurationslogik mit einem passenden intuitiven Bedienkonzept. 


\section{Ziel der Arbeit} \label{goal}
Im Rahmen dieser Arbeit soll eine prototypische Tablet-PC Anwendung erstellt werden. Der Inhalt dieser App besteht zum Einen aus dem vorhandenen Produktkatalog in Papierform und zum Anderen aus einer Anbindung an den Produktkonfigurator.  Die Anwendung muss ein Feedback geben können, ob die derzeitige Auswahl technisch umsetzbar ist. Hierzu soll der vorhandene Konfigurator eines Kunden angebunden werden.
\par
Der Fokus der Anwendung liegt in der Konzeption eines Bedienkonzeptes, welches die vorhandenen Möglichkeiten der Technologie ausnutzt. Der Nutzen für den Kunden soll eine Vereinfachung des derzeitigen Konfigurations-Prozesses mithilfe der Software sein. Ebenfalls sollen durch den mobilen Einsatz der App neue Möglichkeiten bei der Konfiguration entstehen. Der Papierkatalog soll hierfür mit den technologischen Möglichkeiten erweitert und dadurch vereinfacht werden. Für die Zielerreichung ist die  Auswahl der Plattform, sowie die Entscheidung, ob Native- oder Web-Anwendung entscheidend. 


\section{Vorgehensweise}
Am Anfang der Arbeit wird der Workflow des Kunden nachvollzogen. Dieser beinhaltet die Schritte von der Auswahl eines Produktes im Katalog, über die Prüfung der Machbarkeit, bis zur Bestellung des gewählten Elements. Ist der Prozess verstanden, kann darauf aufbauend die Entscheidung der Plattform folgen. Hierbei werden die drei vorhandenen Möglichkeiten mobil, hybrid und nativ untersucht. Ist die Art der Anwendung ausgewählt, kann mit der Entscheidung der Technologie fortgefahren werden. Nach der Auswahl wird die Anwendung konzipiert, wobei die intuitiven Bedienelemente der gewählten Technologie verwendet werden. Nach der Konzeption erfolgt die Implementierung des Prototyps. Anschließend wird eine Evaluation erfolgen, welche die Anwendung nach den zuvor bestimmten Kriterien bewerten soll.

\par
Kapitel \ref{chapter_2} behandelt die grundlegenden Anforderungen der Anwendung und gibt einen Überblick des Produktes Merlin Enterprise, sowie dessen Einsatz beim Kunden . In Kapitel \ref{chapter_3} wird die Konzeption der Anwendung und die darauf resultierende Entscheidung über die passende Plattform behandelt. Anschließend erfolgt in Kapitel \ref{chapter_4} das Entwerfen der einzelnen Ansichten, bevor in den Kapiteln \ref{chapter_5} und \ref{chapter_6} die Implementierung und Evaluation der Anwendung beschrieben wird. Im letzten Abschnitt wird es einen Ausblick und ein Fazit über die gesamte Arbeit geben.


%Um dem Berater ebenfalls ein schnelles Feedback über die technische Realisierung geben zu können, soll es ebenfalls die Möglichkeit geben die Konfiguration zu überprüfen. 

%Diese Anwendung soll den vorhandenen Produktkatalog in Papierform beinhalten, sowie die vorhandene Konfigurationslogik verwenden. Sie soll den Vertriebler beim Verkaufsgespräch unterstützen und im Hintergrund die technische Baubarkeit des Produktes prüfen. Der Fokus der Arbeit soll hierbei an einer ansprechenden optischen Darstellung und einem geeigneten Bediekonzept gelegt werden. Für die Umsetzung ist eine vorige Technologieentscheidung notwendig. 
%Abkürzungen kurz: \acrshort{DHBW},

%Ausgebschriebene Abkürzungen: \gls{DHBW}, 

%Verweise auf das Glossar: \gls{Glossareintrag}, \glspl{Glossareintrag}

%Literaturverweise: \cite{bib:ix042010}, \cite{bib:metasploitBuch}

%\footnote{Ich bin eine Fußnote}
