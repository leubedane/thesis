\chapter{Einleitung}
\section{Motivation} \label{aufgaben}
Die Entwicklung von der Massen-Produktion zur Massen-Individualisierung (engl. mass-customization) bei Produkten schreitet immer weiter voran.\cite{bib:massCustomization}. Mit der höheren Produktvielfalt können auch individuelle Kundenwünsche bedient werden. Bedingt durch die hohe Komplexität, die durch diesen Trend notwendig ist, wird die Zusammenstellung des Produktes aufwändiger. Bisweilen sind die für die Durchführung einer Produkt-Individualisierung viel Zeit, Geld und personelle Ressourcen notwendig.
Nach einer erfolgten Produktauswahl aus einem Produktkatalog in Papierform, wird die Zusammenstellung manuell geprüft, sodass der Kunde ein Feedback über die technische Realisierung erhält. \par

Für die Lösung dieses Problems werden zur Qualitätssteigerung und aus ökonomischen Gesichtspunkten heraus immer mehr computergestützte Systeme verwendet. Diese können innerhalb von Sekunden die Abhängigkeiten der Produkte berechnen und ein schnelles Feedback liefern. Für eine weitere Verbesserung des Prozesses werden bereits mobile Anwendungen auf sogenannten Tablet-PCs konzipiert, welche den Vorteil haben, dass die Lösung beim Kunden vor Ort eingesetzt werden kann. Diese Geräte und deren Anwendungen, Apps genannt, sind im Geschäftsumfeld zunehmend in verschiedenen Bereichen verbreitet\cite{bib:businessApps}. Die Vorteile dieser Systeme sind neben der mobilen Verfügbarkeit eine Verwendung von Touchscreens. Diese bieten eine intuitive Bedienung, womit komplexe Sachverhalte vereinfacht durchgeführt werden können.
 
\section{Ziel der Arbeit} \label{goal}
Im aufgezeigten Rahmen soll die vorliegende Bachelorarbeit eine Möglichkeit aufzeigen, wie eine komplexe Produktlandschaft für einen Kunden übersichtlich dargestellt werden kann. Hierzu wird mithilfe eines Produktkataloges das Produkt aufbereitet und mit einer Konfigurationslösung die Zusammenstellung überprüft. Der Kunde soll hierdurch in der Lage sein, eine Konfiguration selbstständig durchzuführen. 

Für die Unterstützung des Gesamtprozesses wird ein mobiles Endgerät verwendet. Hier soll der Einsatz einer touchgesteuerten Bedienung die Vereinfachung des Prozesses unterstützen. Damit eine bessere Vermittlung des Produktes stattfindet, muss die Anwendung unter Einsatz der technischen Möglichkeiten eine ansprechende Darstellung bieten. 

Bei einem mobilen Einsatz der Anwendung sind Anpassungen an en Konfigurationsprozess nötig. Diese sollen die Effizienz beim Durchführen einer Konfiguration erhöhen. Die Herausforderung besteht in der Umsetzung der angepassten Prozesse als mobile Anwendung. Hier sollen Heuristiken beachtet werden, die eine Integration in die Zielumgebung ermöglicht. 


\section{Vorgehensweise}
Ausgangspunkt der Arbeit ist eine ausgiebige Analyse des Ist-Zustandes eines Konfigurationsprozesses. Dieser Prozess wird für einen neuen mobilen Workflow angepasst. Für die resultierende Arbeit wird ein Anwendungsbeispiel verwendet. An diesem werden die Veränderungen des Prozesses gezeigt. Aus den einzelnen Prozessschritten werden die Anforderungen spezifiziert. Basierend auf dieser Spezifikation wird ein passender Anwendungstyp und Technologie gewählt. Ein Entwurf der benötigten Ansichten erfolgt nach den Vorgaben der Plattform sowie den Anforderungen. Nach der Implementierung erfolgt eine Evaluation der Zielvorgaben. 
\par
\textbf{Abriss: }
Kapitel \ref{chapter_2} beschreibt die Grundlagen der Arbeit. In Kapitel \ref{chapter_3} wird der Prozess und die Anforderungen analysiert. Das Kapitel \ref{chapter_4} behandelt das Entwerfen der einzelnen Ansichten, bevor in Abschnitt \ref{chapter_5} und \ref{chapter_6} die Implementierung und Evaluation der Anwendung beschrieben wird. Zuletzt wird es einen Ausblick und ein Fazit über die gesamte Arbeit geben.



%Abkürzungen kurz: \acrshort{DHBW},

%Ausgebschriebene Abkürzungen: \gls{DHBW}, 

%Verweise auf das Glossar: \gls{Glossareintrag}, \glspl{Glossareintrag}

%Literaturverweise: \cite{bib:ix042010}, \cite{bib:metasploitBuch}

%\footnote{Ich bin eine Fußnote}
