\chapter{Einleitung}
\section{Motivation} \label{aufgaben}
Die Entwicklung von einer Massen-Produktion zu einer Massen-Individualisierung (engl. mass-customization) bei Produkten schreitet immer weiter voran.\cite{bib:massCustomization}. Mit der höheren Produktvielfalt können auch individuelle Kundenwünsche bedient werden. Bedingt durch die hohe Komplexität, die durch diesen Trend notwendig ist, wird eine Zusammenstellung des Produktes aufwändiger. Die Durchführung einer solchen Produkt-Individualisierung erfolgt nach einem gegebenen Workflow. Nach einer erfolgten Produktauswahl aus einem Produktkatalog in Papierform, wird die Zusammenstellung manuell geprüft, sodass der Kunde ein Feedback über die technische Realisierung erhält. Dieser Vorgang kostet viel Zeit, Geld und Kapazitäten innerhalb eines Unternehmens. \par

Für die Lösung dieses Problems werden zur Qualitätssteigerung und aus ökonomischen Gesichtspunkten heraus immer mehr computergestützte Systeme verwendet. Diese können innerhalb von Sekunden die Abhängigkeiten der Produkte berechnen und ein schnelles Feedback liefern. Der nächste Schritt ist eine mobile Verwendung des Systems auf sogenannten Tablet-PCs. Diese Geräte und deren Anwendungen, Apps genannt, finden im Geschäftsumfeld immer mehr Anwendungsfelder \cite{bib:businessApps}. Die Vorteile dieser Systeme sind neben der mobilen Verfügbarkeit eine Verwendung von Touchscreens. Diese bieten eine intuitive Bedienung, womit komplexe Sachverhalte vereinfacht durchgeführt werden können.
 
\section{Ziel der Arbeit} \label{goal}
Die Arbeit soll eine Möglichkeit aufzeigen, wie eine komplexe Produktlandschaft für einen Kunden übersichtlich dargestellt werden kann. Hierzu wird mithilfe eines Produktkataloges das Produkt aufbereitet und mit einer Konfigurationslösung die Zusammenstellung überprüft. Der Kunde soll hierdurch in der Lage sein, eine Konfiguration selbstständig durchzuführen. Dies

Für die Unterstützung des Gesamtprozesses wird ein mobiles Endgerät verwendet. Hier soll der Einsatz einer touchgesteuerten Bedienung die Vereinfachung des Prozesses unterstützen. Weiterhin muss die Anwendung unter Einsatz der technischen Möglichkeiten eine ansprechende Darstellung für eine bessere Vermittlung des Produkts bieten. 

Bei einem mobilen Einsatz der Anwendung muss der Konfigurationsprozess entsprechend angepasst werden. Die Anpassungen sollen die Effizienz beim Durchführen einer Konfiguration erhöhen. Die Herausforderung besteht in der Umsetzung der angepassten Prozesse als mobile Anwendung. Hier sollen Heuristiken beachtet werden, die eine Integration in die Zielumgebung ermöglicht. 



\begin{mdframed}[backgroundcolor=gray!40,shadow=true,roundcorner=8pt]
\textbf{Ziel:} \newline
Vereinfachung des Konfigurationsprozesses eines komplexen Produktes durch den Einsatz einer touchgesteuerten Benutzerschnittstelle.
\end{mdframed}

\section{Vorgehensweise}
Ausgangspunkt der Arbeit ist eine ausgiebige Analyse des Ist-Zustandes eines Kundenprozesses bei der Produktkonfiguration. Hierauf aufbauend werden die Anforderungen der Anwendung spezifiziert. Dies Auswahl einer geeigneten mobilen Plattform erfolgt im nächsten Schritt. Diese werden anhand der spezifizierten Anforderungen gewählt. In Folge der Entscheidung über die Plattform folgt der Entwurf der Ansichten. Die entworfenen Elemente werden im Folgenden bei der Implementierung umgesetzt. Am Ende der Arbeit wird für die Sicherstellung der zuvor gestellten Ziele eine Evaluation der Arbeit. 
\par
\textbf{Abriss: }
Kapitel \ref{chapter_2} werden die Grundlagen der Arbeit behandelt . In Kapitel \ref{chapter_3} wird der Prozess und die Anforderungen analysiert. Das Kapitel \ref{chapter_4} behandelt das Entwerfen der einzelnen Ansichten, bevor in Abschnitt \ref{chapter_5} und \ref{chapter_6} die Implementierung und Evaluation der Anwendung beschrieben wird. Zuletzt wird es einen Ausblick und ein Fazit über die gesamte Arbeit geben.



%Abkürzungen kurz: \acrshort{DHBW},

%Ausgebschriebene Abkürzungen: \gls{DHBW}, 

%Verweise auf das Glossar: \gls{Glossareintrag}, \glspl{Glossareintrag}

%Literaturverweise: \cite{bib:ix042010}, \cite{bib:metasploitBuch}

%\footnote{Ich bin eine Fußnote}
