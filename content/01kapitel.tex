\chapter{Einleitung}
\section{Motivation} \label{aufgaben}
Die Entwicklung von einer Massen-Produktion zu einer Massen-Individualisierung (engl. mass-customization) bei Produkten schreitet immer weiter voran.\cite{bib:massCustomization}. Mit der höheren Produktvielfalt können auch individuelle Kundenwünsche bedient werden. Bedingt durch die hohe Komplexität, die durch diesen Trend notwendig ist, wird eine Zusammenstellung des Produktes aufwändiger. Die Durchführung einer solchen Produkt-Individualisierung erfolgt nach einem gegebenen Workflow. Nach einer erfolgten Produktauswahl aus einem Produktkatalog in Papierform, wird die Zusammenstellung manuell geprüft, sodass der Kunde ein Feedback über die technische Realisierung erhält. Dieser Vorgang kostet viel Zeit, Geld und Kapazitäten innerhalb eines Unternehmens. \par

Für die Lösung dieses Problems werden zur Qualitätssteigerung und aus ökonomischen Gesichtspunkten heraus immer mehr computergestützte Systeme verwendet. Diese können innerhalb von Sekunden die Abhängigkeiten der Produkte berechnen und ein schnelles Feedback liefern. Der Vertrieb eines Produktes sollte möglichst nahe beim Kunden durchgeführt werden. Aus diesem Grund müssen Produktkataloge eine hohe Verfügbarkeit, sowie eine einfache Handhabung aufweisen. Deshalb sind Produktkataloge optimaler Weise mobil verfügbar. Die Mobilität wird durch das Verwenden sogenannter Apps auf mobilen Endgeräten erreicht. Der Nutzen solcher Apps im Geschäftsumfeld wird immer mehr erkannt \cite{bib:businessApps}. Der Anwender muss bei einer mobilen Anwendung im Zentrum stehen \cite{bib:businessAppsDesign}. Dieses nutzerzentrierte Design wird durch eine hohe Gebrauchstauglichkeit, sowie einen hohen Nutzen für den Anwender erreicht. Aus diesem Grund muss die Anwendung ein durchdachtes Konzept bei der Bedienung beinhalten.


\section{Ziel der Arbeit} \label{goal}
Die Arbeit soll eine Möglichkeit aufzeigen, wie eine komplexe Produktlandschaft für den Kunden übersichtlich dargestellt werden kann. Hierzu sollen komplexe Abhängigkeiten der einzelnen Produkte im Hintergrund von einem Produktkonfigurator berechnet werden. Die Ergebnisse werden dem Benutzer auf eine einfache, verständliche Weise dargestellt. Durch die Umsetzung der Anwendung sollen die Prozesse der Produktkonfiguration auf das Wesentliche, die Produkte, konzentriert werden. Hierzu müssen Konzepte entwickelt werden, die den Benutzer in den Vordergrund stellt, um dessen Bedürfnisse am Besten gerecht zu werden. Für eine Bessere Integration dieses neuen Ansatzes wird der vorhandene Workflow ebenfalls überarbeitet und an die neue Zielsetzung angepasst. 

Damit die resultierende Anwendung einen deutlichen Mehrwert erzielt, muss eine hohe Usability erreicht werden. Diese wird durch ein intuitives und damit einfach zu erlernendes Bedienkonzept erreicht. Als weitere Maßnahme muss eine geeignete Form der Anwendung gewählt werden, die allen Anforderungen entspricht. Hierzu soll eine ausgiebige Analyse der aktuellen Möglichkeiten durchgeführt werden.



\section{Vorgehensweise}
Ausgangspunkt der Arbeit ist eine ausgiebige Analyse des Ist-Zustandes eines Kundenprozesses. Hierauf aufbauend werden die Anforderungen der Anwendung spezifiziert.
Am Anfang der Arbeit wird der Workflow des Kunden nachvollzogen. Dieser beinhaltet die Schritte von der Auswahl eines Produktes im Katalog, über die Prüfung der Machbarkeit, bis zur Bestellung des gewählten Elements. Ist der Prozess verstanden, kann darauf aufbauend die Entscheidung der Plattform folgen. Hierbei werden die drei vorhandenen Möglichkeiten mobil, hybrid und nativ untersucht. Ist die Art der Anwendung ausgewählt, kann mit der Entscheidung der Technologie fortgefahren werden. Nach der Auswahl wird die Anwendung konzipiert, wobei die intuitiven Bedienelemente der gewählten Technologie verwendet werden. Nach der Konzeption erfolgt die Implementierung des Prototyps. Anschließend wird eine Evaluation erfolgen, welche die Anwendung nach den zuvor bestimmten Kriterien bewerten soll.

\par
Kapitel \ref{chapter_2} behandelt die grundlegenden Anforderungen der Anwendung und gibt einen Überblick des Produktes Merlin Enterprise, sowie dessen Einsatz beim Kunden . In Kapitel \ref{chapter_3} wird die Konzeption der Anwendung und die darauf resultierende Entscheidung über die passende Plattform behandelt. Anschließend erfolgt in Kapitel \ref{chapter_4} das Entwerfen der einzelnen Ansichten, bevor in den Kapiteln \ref{chapter_5} und \ref{chapter_6} die Implementierung und Evaluation der Anwendung beschrieben wird. Im letzten Abschnitt wird es einen Ausblick und ein Fazit über die gesamte Arbeit geben.


%Um dem Berater ebenfalls ein schnelles Feedback über die technische Realisierung geben zu können, soll es ebenfalls die Möglichkeit geben die Konfiguration zu überprüfen. 

%Diese Anwendung soll den vorhandenen Produktkatalog in Papierform beinhalten, sowie die vorhandene Konfigurationslogik verwenden. Sie soll den Vertriebler beim Verkaufsgespräch unterstützen und im Hintergrund die technische Baubarkeit des Produktes prüfen. Der Fokus der Arbeit soll hierbei an einer ansprechenden optischen Darstellung und einem geeigneten Bediekonzept gelegt werden. Für die Umsetzung ist eine vorige Technologieentscheidung notwendig. 
%Abkürzungen kurz: \acrshort{DHBW},

%Ausgebschriebene Abkürzungen: \gls{DHBW}, 

%Verweise auf das Glossar: \gls{Glossareintrag}, \glspl{Glossareintrag}

%Literaturverweise: \cite{bib:ix042010}, \cite{bib:metasploitBuch}

%\footnote{Ich bin eine Fußnote}
