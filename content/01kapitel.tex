\chapter{Einleitung}
\section{Motivation} \label{aufgaben}
Die Entwicklung von einer Massen-Produktion zu einer Massen-Individualisierung (engl. mass-customization) bei Produkten schreitet immer weiter voran.\cite{bib:massCustomization}. Mit der höheren Produktvielfalt können auch individuelle Kundenwünsche bedient werden. Bedingt durch die hohe Komplexität, die durch diesen Trend notwendig ist, wird eine Zusammenstellung des Produktes aufwändiger. Die Durchführung einer solchen Produkt-Individualisierung erfolgt nach einem gegebenen Workflow. Nach einer erfolgten Produktauswahl aus einem Produktkatalog in Papierform, wird die Zusammenstellung manuell geprüft, sodass der Kunde ein Feedback über die technische Realisierung erhält. Dieser Vorgang kostet viel Zeit, Geld und Kapazitäten innerhalb eines Unternehmens. \par

Für die Lösung dieses Problems werden zur Qualitätssteigerung und aus ökonomischen Gesichtspunkten heraus immer mehr computergestützte Systeme verwendet. Diese können innerhalb von Sekunden die Abhängigkeiten der Produkte berechnen und ein schnelles Feedback liefern. Der nächste Schritt ist eine mobile Verwendung des Systems auf sogenannten Tablet-PCs. Diese Geräte und deren Anwendungen, Apps genannt, finden im Geschäftsumfeld immer mehr Anwendungsfelder \cite{bib:businessApps}. Für die Verwendung dieser Lösungen ist eine Vereinfachung, bzw. Anpassung der Geschäftsprozesse notwendig.

\section{Ziel der Arbeit} \label{goal}
Die Arbeit soll eine Möglichkeit aufzeigen, wie eine komplexe Produktlandschaft für den Kunden übersichtlich dargestellt werden kann. Hierzu sollen komplexe Abhängigkeiten der einzelnen Produkte im Hintergrund von einem Produktkonfigurator berechnet werden. Die Ergebnisse werden dem Benutzer auf eine einfache, verständliche Weise dargestellt. Durch die Umsetzung der Anwendung sollen die Prozesse der Produktkonfiguration auf das Wesentliche, die Produkte, konzentriert werden. Hierzu müssen Konzepte entwickelt werden, die den Benutzer in den Vordergrund stellt, um dessen Bedürfnisse am Besten gerecht zu werden. Für eine Bessere Integration dieses neuen Ansatzes wird der vorhandene Workflow ebenfalls überarbeitet und an die neue Zielsetzung angepasst. \par

Damit die resultierende Anwendung einen deutlichen Mehrwert erzielt, muss eine hohe Usability erreicht werden. Diese wird durch ein intuitives und damit einfach zu erlernendes Bedienkonzept erreicht. Als weitere Maßnahme muss eine geeignete Form der Anwendung gewählt werden, die allen Anforderungen entspricht. Hierzu soll eine ausgiebige Analyse der aktuellen Möglichkeiten durchgeführt werden.

\begin{mdframed}[backgroundcolor=gray!40,shadow=true,roundcorner=8pt]
\textbf{Ziel:} \newline
Vereinfachung eines Konfigurationsprozesses eines komplexen Produktes durch den Einsatz einer touchgesteuerten Oberfläche.
\end{mdframed}

\section{Vorgehensweise}
Ausgangspunkt der Arbeit ist eine ausgiebige Analyse des Ist-Zustandes eines Kundenprozesses bei der Produktkonfiguration. Hierauf aufbauend werden die Anforderungen der Anwendung spezifiziert. Dies Auswahl einer geeigneten mobilen Plattform erfolgt im nächsten Schritt. Diese werden anhand der spezifizierten Anforderungen gewählt. In Folge der Entscheidung über die Plattform folgt der Entwurf der Ansichten. Die entworfenen Elemente werden im Folgenden bei der Implementierung umgesetzt. Am Ende der Arbeit wird für die Sicherstellung der zuvor gestellten Ziele eine Evaluation der Arbeit. 
\par
\textbf{Abriss: }
Kapitel \ref{chapter_2} werden die Grundlagen der Arbeit behandelt . In Kapitel \ref{chapter_3} wird der Prozess und die Anforderungen analysiert. Das Kapitel \ref{chapter_4} behandelt das Entwerfen der einzelnen Ansichten, bevor in Abschnitt \ref{chapter_5} und \ref{chapter_6} die Implementierung und Evaluation der Anwendung beschrieben wird. Zuletzt wird es einen Ausblick und ein Fazit über die gesamte Arbeit geben.


%Um dem Berater ebenfalls ein schnelles Feedback über die technische Realisierung geben zu können, soll es ebenfalls die Möglichkeit geben die Konfiguration zu überprüfen. 

%Diese Anwendung soll den vorhandenen Produktkatalog in Papierform beinhalten, sowie die vorhandene Konfigurationslogik verwenden. Sie soll den Vertriebler beim Verkaufsgespräch unterstützen und im Hintergrund die technische Baubarkeit des Produktes prüfen. Der Fokus der Arbeit soll hierbei an einer ansprechenden optischen Darstellung und einem geeigneten Bediekonzept gelegt werden. Für die Umsetzung ist eine vorige Technologieentscheidung notwendig. 
%Abkürzungen kurz: \acrshort{DHBW},

%Ausgebschriebene Abkürzungen: \gls{DHBW}, 

%Verweise auf das Glossar: \gls{Glossareintrag}, \glspl{Glossareintrag}

%Literaturverweise: \cite{bib:ix042010}, \cite{bib:metasploitBuch}

%\footnote{Ich bin eine Fußnote}
