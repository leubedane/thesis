

\chapter{Grundlagen} \label{chapter_2}

In diesem Kapitel werden die gezielten Anforderungen der Arbeit geschildert, die Anwendungsdomäne erläutert und der Konfigurator Merlin vorgestellt. 

\section{Grundlegende Anforderungen}
Die folgenden Anforderungen wurden nach Rücksprache mit Entwicklern und dem Projektleiter zusammen erarbeitet. Zusammen mit den Anforderungen wurde eine Priorisierung der einzelnen Aufgaben durchgeführt. Die Unterteilung der Anforderungen erfolgt in Fachlich und Nicht-Fachlich.

\subsection{Nicht-Fachliche Anforderungen}
\begin{tabular}{| p{1.1cm} | p{2.2cm} | p{4.3cm} | p{4.9cm} | p{1.3cm} |}
\toprule[2pt] \rowcolor{dunkelgrau}
\hline
  Kürzel & Anforderung & Beschreibung & Details & Priorität \\
  \hline
  B1 & Einfache \newline Bedienung & Die Anwendung soll von unerfahrenen Benutzern schnell bedient werden können.& Schwerpunkte der Softwareergonomie\cite{bib:softwareErgonomie}: 
  \begin{itemize}
        \item Selbstbeschreibungs-fähigkeit
        \item Lernförderlichkeit
        \item Erwartungs-konformität
     \end{itemize}
   & A \\
  \hline
  B2 & Schnelle Navigation & Dem Benutzer soll die Möglichkeit gegeben werden schnell zwischen den einzelnen Bereichen zu navigieren. & Es muss eine zentrale Navigationsmöglichkeit geben, die jederzeit erreichbar ist. Diese ermöglicht einen schnellen Wechsel zwischen den Ansichten. & A \\
    \hline
\bottomrule[2pt]
 \end{tabular}

\section{Abgrenzung der Arbeit}

\section{Umfeld der Arbeit}

\section{CAS Configurator Merlin}
