\chapter{Evaluation der Anwendung}\label{chapter_6}
Damit ein Kunde einen besseren Einblick in das Produkt erhält, muss der Konfigurationsprozess vereinfacht werden. Die Vereinfachung des Prozesses wird ebenfalls durch eine einfache Bedienung der Anwendung erreicht. Die Sicherstellung, dass dieses Ziel bei der Entwicklung der App erreicht wurde, soll anhand einer heuristischen Evaluation herausgefunden werden. Hierbei wird eine bestimmte Anzahl an Benutzern herausgesucht, die eine Untersuchung der Anwendung durchführen. Die Heuristiken der einzelnen Benutzer ergeben sich aus bekannten Benutzungsprinzipien (siehe: \cite{bib:heuristik1}). Die Ergebnisse der Evaluation werden anhand der 10 Heuristiken von Nielsen \cite{bib:heuristik2} ausgewertet.


\section{Durchführung der Evaluation}
Für die Umstellung des Prozesses, wie in Kapitel \ref{chapter_3} beschrieben muss die Anwendung den neuen Prozess unterstützten. Hier muss die Evaluation die Frage beantworten, ob die Produktauswahl, sowie der Konfigurationsprozess verständlich ist. Damit ein Unterschied festgestellt werden kann, werden bestimmte Aufgaben der Anwendung zuerst auf der alten Konfigurationsoberfläche durchgeführt und anschließend mit der neuen Anwendung. Die Durchführung erfolgt hierbei in einer Interview Form, bei der eine Hilfestellung bei der Bedienung gegeben wird. Anschließend erhält der Anwender einen Fragebogen, der im Anhang zu finden ist. Mit diesem Vorgang erhält man zwei Rückmeldungen. Die Erste durch die verschiedenen Fragen der Benutzer bei der Anwendung und die zweite durch den ausgefüllten Fragebogen. Für eine aussagekräftige Evaluation wurden 4 Testpersonen ausgewählt. Zusätzlich zu der Benutzerevaluation wird eine zweite Evaluation der Experten durchgeführt. Hier wurde anhand der 10 Heuristiken ein Fragebogen angefertigt, der alle Merkmale enthält. Der Experte soll durch beliebiges Verwenden der Anwendung die Fragen bewerten. 


\section{Ergebnisse}

\subsection{Allgemeine Auswertung}
\subsection{Benutzerauswertung}
\subsection{Expertenauswertung}
