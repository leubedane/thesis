\chapter{Fazit und Ausblick}\label{chapter_7}
In der vorliegenden Arbeit wurde ein Weg gezeigt, wie ein komplexes Produkt übersichtlich dargestellt werden kann. Nach einer Einführung in die Thematik und die Anwendungsdomäne wurde eine ausgiebige Analyse des Ist-Zustandes eines Konfigurationsprozesses durchgeführt. Die aufgetretenen Probleme sind in einem neuen Prozess behoben worden, der zusätzliche Anpassungen aufgrund der mobilen Zielumgebung enthält. Aus dem neuen Workflow sind die Anforderungen der Anwendung spezifiziert worden. Diese sind für eine Auswahl des passenden mobilen Anwendungstypen, sowie der entsprechenden Technologie verwendet worden. Nach der Auswahl der Plattform sind die einzelnen Prozessschritte für die mobile App konzipiert worden. Die folgende Implementierung wurde mit einem passenden Architektur-Pattern umgesetzt. Eine Evaluation am Ende hat die Erfüllung der Ziele bestätigt und konnte weitere Verbesserungsmöglichkeiten aufzeigen.

Die zugrundeliegenden Komplexität des Produktes wurde durch eine übersichtliche Aufbereitung des Inhalts in einer mobilen, touchgesteuerten Umgebung erreicht. Hierbei wurden besondere Bedienelemente der ausgewählten Technologie für eine einfache und schnelle Bedienung erfolgreich eingesetzt. Der Kunde kann die Anwendung damit genauso bedienen wie ein Mitarbeiter des Herstellers. 

Innerhalb der Arbeit ist eine Verbindung zwischen einem Produktkatalog und einer Konfigurationslösung zu einer gemeinsamen Komponente entstanden.  Durch die Software aus einer Hand können die Geschäftsprozesse für eine Individualisierung des Produktes schneller durchgeführt werden und der Kunde erhält ein schnelleres Feedback zu seiner Auswahl. Die App ist somit für eine Vereinfachung des Konfigurationsprozesses bereit und kann einen ökonomischen und komfortablen Weg der Konfiguration bieten, wodurch ein deutlicher Mehrwert beim Einsatz entsteht.  \par 

Der neue mobile Workflow wurde für ein konkretes Beispiel aus der Luftfahrt umgesetzt. Für die Zukunft ist eine Anwendung des neuen Prozesses auf weitere Geschäftsfelder interessant. Voraussetzung für die Umsetzung ist ein komplexes Produkt mit vielen Abhängigkeiten der einzelnen Bauteile, die für einen Kunden verständlich dargestellt werden müssen. Aus der Umsetzung auf einem weiteren Gebiet kann der modellierte Prozess weiter verbessert werden, damit die Effizienz bei der Durchführung einer Konfiguration für die Individualisierung eines Produktes weiter steigt.