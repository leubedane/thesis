\chapter{Fazit und Ausblick}\label{chapter_7}
Bei der vorliegenden Arbeit wurde ein Weg gezeigt, wie ein komplexes Produkt anschaulich dargestellt werden kann. Die zugrundeliegenden Komplexität wurde durch eine übersichtliche Aufbereitung des Inhalts in einer mobilen, touchgesteuerten Umgebung erreicht. Durch die implementierte Anwendung kann eine Vereinfachung des Konfigurationsprozesses mit einem Produktkonfigurator erfolgen. Innerhalb der Arbeit wurde eine Verbindung zwischen einem Produktkatalog und einer Konfigurationslösung zu einer gemeinsamen Komponente erreicht.  Durch die Software aus einer Hand können die Geschäftsprozesse für eine Individualisierung des Produktes schneller durchgeführt werden und der Kunde erhält ein schnelleres Feedback zu seiner Auswahl. \par 

Beim Entwurf der Arbeit wurden die Besonderheiten der mobilen Zielumgebung genauso berücksichtigt, wie die Vorgaben der verwendeten Technologie. Hierdurch wurde eine auf die mobile Umgebung angepasste Anwendung entwickelt. Besonders gelungen ist die Implementierung der Entwürfe. Diese konnten detailgetreu umgesetzt werden, so dass ein überzeugendes Endergebnis entstanden ist. Mit einer abschließenden Evaluation wurden die Funktionalen und Nicht-Funktionalen Anforderungen der Anwendung überprüft. Das Ergebnis hat gezeigt, dass die gestellten Kriterien erfüllt worden sind. Weiteres Verbesserungspotenzial wurde ebenfalls erkannt, aufgrund dessen ein neuer Entwurf der betroffenen Ansicht resultierte. 

Für eine zukünftige Entwicklung der App sind die weiteren Heuristiken von Nielsen interessant. Diese waren im Evaluationsbogen der Experten, jedoch nicht in der Zielsetzung der Anforderungen enthalten. Aufgrund der unterschiedlichen Bewertung der Experten in diesen Kriterien können hier weitere Verbesserungen durchgeführt werden. \par 

Die App ist für eine Vereinfachung des Konfigurationsprozesses bereit und kann einen ökonomischen und komfortablen Weg der Konfiguration bieten, wodurch ein deutlicher Mehrwert beim Einsatz entsteht.

