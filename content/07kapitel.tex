\chapter{Fazit und Ausblick}\label{chapter_7}
Bei der vorliegenden Arbeit wurde ein Weg gezeigt, wie ein komplexes Produkt anschaulich dargestellt werden kann. Das Aufbereiten der zugrundeliegenden Komplexität wurde durch eine anschauliche Umsetzung der Inhalte in einer mobilen, touchgesteuerten Umgebung erreicht. Durch diese Voraussetzung kann eine Vereinfachung des Konfigurationsprozesses mit einem Produktkonfigurator erfolgen. Innerhalb der Arbeit wurde eine Verbindung zwischen dem Produktkatalog und dem Konfigurator zu einer gemeinsamen Lösung erreicht.  Durch die Software aus einer Hand können die Geschäftsprozesse für eine Individualisierung des Produktes schneller durchgeführt werden und der Kunde erhält ein schnelleres Feedback zu seiner Auswahl. \par 

Beim Entwurf der Arbeit wurden die Besonderheiten der mobilen Zielumgebung genauso berücksichtigt, wie die Vorgaben der verwendeten Technologie. Hierdurch wurde eine auf die mobile Umgebung angepasste Anwendung entwickelt, die bei der anschließenden Evaluation überzeugen konnte. Die Auswertung der Fragebögen hat Probleme bei der Anwendung erkannt, die durch einen neuen Entwurf und Umsetzung der betroffenen Seite gelöst werden konnten. Bei der Implementierung konnte durch die Verwendung eines passenden Entwurfsmusters eine zukünftige Weiterentwicklung der App ermöglicht werden. Die Umsetzung des neuen Workflows wird sich beim Einsatz der Lösung bei einem Kunden zeigen. 