%%**************************************************************
%% Vorlage fuer Bachelorarbeiten (o.ä.) der DHBW
%%
%% Autor: Tobias Dreher, Yves Fischer
%% Datum: 06.07.2011
%%**************************************************************

\newcommand{\pdftitel}{Bachelor Thesis}
\newcommand{\autor}{Dane Leube}
\newcommand{\arbeit}{Bachelorthesis}
 
\input{header} 

% Ab jetzt können auch Umlaute verwendet werden
\newcommand{\titel}{Konzeption und Implementierung einer touchgesteuerten Oberfläche für einen
konfiguratorbasierten Produktkatalog
}
\newcommand{\martrikelnr}{1313394}
\newcommand{\kurs}{TAI10B2}
\newcommand{\datumAbgabe}{27.03.2013}
\newcommand{\firma}{CAS Software AG}
\newcommand{\firmenort}{Karlsruhe}
\newcommand{\abgabeort}{Karlsruhe}
\newcommand{\abschluss}{Bachelor of Science}
\newcommand{\studiengang}{Studienganges Angewandte Informatik}
\newcommand{\dhbw}{Karlsruhe}
\newcommand{\betreuer}{Dr. Michael Klein}
\newcommand{\gutachter}{Dipl. Inform. Thorsten Schlachter}
\newcommand{\zeitraum}{12 Wochen}

\makeglossaries
%
% vorher in Konsole folgendes aufrufen: 
%	makeglossaries makeglossaries dokumentation.acn && makeglossaries dokumentation.glo
%

%
% Abkürzungen --> referenz, name, beschreibung
% Aufruf mit \gls{...} oder Kurzform mit \acrshort{...}
%

\newacronym{APP}{APP}{Kurzform für Applikation/Anwendung}
\newacronym{RCP}{RCP}{Rich Client Plattform}
\newacronym{PKS}{PKS}{Konfigurations-Server}
\newacronym{PK-Client}{PK-Client}{Produkt Konfigurator Client (Swing basiert)}
\newacronym{PMS}{PMS}{Produktmanagementsystem}
\newacronym{MVVM}{MVVM}{Model View ViewModel}
\newacronym{TDD}{TDD}{Test Driven Development (Test getriebene Entwicklung)}

%
% Glossareintraege --> referenz, name, beschreibung
% Aufruf mit \gls{...}
%
\newglossaryentry{Glossareintrag}{name={Glossareintrag},plural={Glossareinträge},description={Ein Glossar beschreibt verschiedenste Dinge in kurzen Worten}}


\begin{document}

	% Deckblatt
	\begin{spacing}{1}
		\input{deckblatt}
	\end{spacing}
	\newpage
	
	
	\renewcommand{\thepage}{\Roman{page}}
	\setcounter{page}{1}	
	
	% Erklärung
	\input{erklaerung}
	\newpage

	% Abstract
	\thispagestyle{empty}

\renewcommand{\abstractname}{abstract}
\begin{abstract}
Das Voranschreiten der sogenannten mass-customization bei Produkten erfordert immer komplexere Produkte, um ein hohes Maß an Individualität erreichen zu können. Für den Kunden im Mittelpunkt steht die Auswahl der einzelnen Komponenten des Produktes. Die Komplexität der Produkte soll für den Kunden nicht sichtbar sein. Es muss somit ein Weg gefunden werden, wie eine komplexe Produktlandschaft für den Kunden vereinfacht dargestellt werden kann. Im Rahmen dieser Bachelorthesis wird mit der Umsetzung eines Produktkataloges auf eine mobile Zielumgebung versucht dieses Ziel zu erreichen. Hierbei werden die wichtigsten Bedürfnisse des Kunden analysiert und darauf aufbauend eine Anwendung konzipiert. Durch die optimierte Darstellung der einzelnen Produkte, sowie eine Überprüfung der Zusammenstellung im Hintergrund wird ein Mehrwert für den Kunden erzielt. Die Sicherstellung der Zielerreichung wurde durch eine Evaluation der Lösung erreicht.

\end{abstract}


%\renewcommand{\abstractname}{Summary}
%\begin{abstract}
%An abstract is a brief summary of a research article, thesis, review,
%conference proceeding or any in-depth analysis of a particular subject
%or discipline, and is often used to help the reader quickly ascertain
%the paper's purpose. When used, an abstract always appears at the
%beginning of a manuscript, acting as the point-of-entry for any given
%scientific paper or patent application. Abstracting and indexing
%services for various academic disciplines are aimed at compiling a
%body of literature for that particular subject.

%The terms précis or synopsis are used in some publications to refer to
%the same thing that other publications might call an "abstract". In
%management reports, an executive summary usually contains more
%information (and often more sensitive information) than the abstract
%does.

%Quelle: \url{http://en.wikipedia.org/wiki/Abstract_(summary)}

%\end{abstract}

	\newpage

	% Inhaltsverzeichnis
	\begin{spacing}{1.1}
		\setcounter{tocdepth}{2}
		\tableofcontents
	\end{spacing}
	\newpage
	\renewcommand{\thepage}{\arabic{page}}
	\setcounter{page}{1}
	
	% Inhalt\part{title}
	\chapter{Einleitung}
\section{Motivation} \label{aufgaben}
Die Entwicklung von der Massen-Produktion zur Massen-Individualisierung (engl. mass-customization) bei Produkten schreitet immer weiter voran.\cite{bib:massCustomization}. Mit der höheren Produktvielfalt können auch individuelle Kundenwünsche bedient werden. Bedingt durch die hohe Komplexität, die durch diesen Trend notwendig ist, wird die Zusammenstellung des Produktes aufwändiger. Bisweilen sind die für die Durchführung einer Produkt-Individualisierung viel Zeit, Geld und personelle Ressourcen notwendig.
Nach einer erfolgten Produktauswahl aus einem Produktkatalog in Papierform, wird die Zusammenstellung manuell geprüft, sodass der Kunde ein Feedback über die technische Realisierung erhält. \par

Für die Lösung dieses Problems werden zur Qualitätssteigerung und aus ökonomischen Gesichtspunkten heraus immer mehr computergestützte Systeme verwendet. Diese können innerhalb von Sekunden die Abhängigkeiten der Produkte berechnen und ein schnelles Feedback liefern. Für eine weitere Verbesserung des Prozesses werden bereits mobile Anwendungen auf sogenannten Tablet-PCs konzipiert, welche den Vorteil haben, dass die Lösung beim Kunden vor Ort eingesetzt werden kann. Diese Geräte und deren Anwendungen, Apps genannt, sind im Geschäftsumfeld zunehmend in verschiedenen Bereichen verbreitet\cite{bib:businessApps}. Die Vorteile dieser Systeme sind neben der mobilen Verfügbarkeit eine Verwendung von Touchscreens. Diese bieten eine intuitive Bedienung, womit komplexe Sachverhalte vereinfacht durchgeführt werden können.
 
\section{Ziel der Arbeit} \label{goal}
Im aufgezeigten Rahmen soll die vorliegende Bachelorarbeit eine Möglichkeit aufzeigen, wie eine komplexe Produktlandschaft für einen Kunden übersichtlich dargestellt werden kann. Hierzu wird mithilfe eines Produktkataloges das Produkt aufbereitet und mit einer Konfigurationslösung die Zusammenstellung überprüft. Der Kunde soll hierdurch in der Lage sein, eine Konfiguration selbstständig durchzuführen. 

Für die Unterstützung des Gesamtprozesses wird ein mobiles Endgerät verwendet. Hier soll der Einsatz einer touchgesteuerten Bedienung die Vereinfachung des Prozesses unterstützen. Damit eine bessere Vermittlung des Produktes stattfindet, muss die Anwendung unter Einsatz der technischen Möglichkeiten eine ansprechende Darstellung bieten. 

Bei einem mobilen Einsatz der Anwendung sind Anpassungen an en Konfigurationsprozess nötig. Diese sollen die Effizienz beim Durchführen einer Konfiguration erhöhen. Die Herausforderung besteht in der Umsetzung der angepassten Prozesse als mobile Anwendung. Hier sollen Heuristiken beachtet werden, die eine Integration in die Zielumgebung ermöglicht. 



\begin{mdframed}[backgroundcolor=gray!40,shadow=true,roundcorner=8pt]
\textbf{Ziel:} \newline
Vereinfachung des Konfigurationsprozesses eines komplexen Produktes durch den Einsatz einer touchgesteuerten Benutzerschnittstelle.
\end{mdframed}

\section{Vorgehensweise}
Ausgangspunkt der Arbeit ist eine ausgiebige Analyse des Ist-Zustandes eines Konfigurationsprozesses. Dieser Prozess wird für einen neuen mobilen Workflow angepasst. Für die resultierende Arbeit wird ein Anwendungsbeispiel verwendet. An diesem werden die Veränderungen des Prozesses gezeigt. Aus den einzelnen Prozessschritten werden die Anforderungen spezifiziert. Basierend auf dieser Spezifikation wird ein passender Anwendungstyp und Technologie gewählt. Ein Entwurf der benötigten Ansichten erfolgt nach den Vorgaben der Plattform sowie den Anforderungen. Nach der Implementierung erfolgt eine Evaluation der Zielvorgaben. 
\par
\textbf{Abriss: }
Kapitel \ref{chapter_2} beschreibt die Grundlagen der Arbeit. In Kapitel \ref{chapter_3} wird der Prozess und die Anforderungen analysiert. Das Kapitel \ref{chapter_4} behandelt das Entwerfen der einzelnen Ansichten, bevor in Abschnitt \ref{chapter_5} und \ref{chapter_6} die Implementierung und Evaluation der Anwendung beschrieben wird. Zuletzt wird es einen Ausblick und ein Fazit über die gesamte Arbeit geben.



%Abkürzungen kurz: \acrshort{DHBW},

%Ausgebschriebene Abkürzungen: \gls{DHBW}, 

%Verweise auf das Glossar: \gls{Glossareintrag}, \glspl{Glossareintrag}

%Literaturverweise: \cite{bib:ix042010}, \cite{bib:metasploitBuch}

%\footnote{Ich bin eine Fußnote}

	\chapter{Grundlagen} \label{chapter_2}

\section{Grundlegende Anforderungen}
\section{Abgrenzung der Arbeit}

\section{Umfeld der Arbeit}

\section{CAS Configurator Merlin}

	\chapter{Analyse der Anwendung}\label{chapter_3}
Damit der Konfigurationsprozess für den Kunden einfacher wird, muss im ersten Schritt der aktuelle Prozess analysiert werden. Darauf aufbauend werden Konzepte zur Vereinfachung dieses Prozesses überlegt und ein neuer Workflow erstellt. Die Anforderungen an die neue Anwendung werden anschließend spezifiziert.

\section{Aktueller Konfigurationsprozess}
Die Anpassungen und Vereinfachungen eines Konfigurationsprozesses bei der Verwendung von mobilen Anwendungen sollen am konkreten Anwendungsbeispiel gezeigt werden. Die Analyse beginnt bei grundlegenden Fragen, wie ein solcher Prozess aufgebaut ist und wird im Folgenden anhand eines konkreten Kundenfalls weiter ausgebaut.

\begin{figure}
\label{oldWorkflow}
\centering
\includegraphics[width=\hsize]{images/konfigurationsprozess_alt}
\caption{Auszug eines Konfigurationsprozesses}
\end{figure}
Ein Auszug aus einem Konfigurationsprozess ist in Abbildung  \ref{oldWorkflow} zu sehen. Die Darstellung zeigt eine Vereinfachung, da es in der Praxis meist Schleifen gibt. Zu Beginn möchte der Kunde ein konkretes Produkt auswählen. Dies geschieht über den Katalog.  Anschließend kann er mithilfe von eindeutigen Produktnummern die Bestellung über ein Formular weitergeben. Damit die Zusammenstellung des Kunden geprüft werden kann, werden die Daten beim Hersteller in das Konfigurationssystem eingepflegt. Dieses berechnet die komplexen Abhängigkeiten und prüft die Gültigkeit der Konfiguration. Der letzte Schritt ist die Angebotserstellung. Hierbei wird das konkrete Angebot für den Kunden erstellt.
\par

Bei der Verwendung dieser Prozessbeschreibung entsteht durch den gegebenen Prozess ein Kommunikationsproblem. Dies tritt aufgrund der Trennung von Produktauswahl und Produktkonfiguration auf. Der Kunde ist nur bei der Auswahl beteiligt. Das Feedback für die Umsetzung der Zusammenstellung erfolgt erst nach der Weitergabe der Bestellung. Dies ist problematisch, wenn bei der Konfiguration Alternativen auftreten, bei denen der Kunde entscheiden muss oder eine Konfiguration komplett nicht umsetzbar ist. Hier muss ggf. der Prozess wiederholt werden. Für die Lösung ist eine Hinzunahme des Kunden bei der Konfigurationsüberprüfung notwendig. Das Ziel sollte ein schnelles Feedback bei der aktuellen Auswahl sein. \par 

Die zweite Stelle, an der eine Verbesserung möglich ist, befindet sich bei der Eingabe der Daten für den Produktkonfigurator und der damit verbundenen Prüfung. Der Kunde hat im vorigen Schritt bereits die Daten, die benötigt werden erfasst. Eine zweite Erfassung kann zu Fehlern oder Kommunikationsproblemen führen. Diese Probleme entstehen bei der falschen Eingabe des bestellten Produktes. Hierdurch kann der Konfigurator ein falsches Ergebnis liefern, welches zu einem inkorrekten Angebot führt, wodurch der Prozess wiederholt werden muss. Für die Verbesserung dieser Situation muss die Eingabe des Kunden direkt zum Konfigurator gegeben werden. Diese Maßnahme minimiert die Fehler bei der Erfassung. \par 

Ein weiteres Problem ist die Verwendung von Produktnummern. Diese Nummern werden für eine eindeutige Identifizierung des Produktes bei der Bestellung verwendet. Bei der Verwendung können Fehler entstehen. Es kann die falsche Produktnummer vom Kunden bei der Auswahl verwendet werden, so dass nicht das richtige Produkt bei der Angebotserstellung verwendet wird. Bei der Produktauswahl sind 
diese Informationen nicht notwendig, da der Kunde sich nur für das konkrete Produkt interessiert. Für die Vereinfachung des Prozesses ist eine automatische Zuordnung der korrekten Nummer im Hintergrund die Lösung. Somit wird der direkte Umgang des Kunden mit einzelnen Produktnummern vermieden und die alleinige Auswahl des Produktes steht für den Kunden im Vordergrund. Hierdurch wird der Prozess vereinfacht. \par



Zusammenfassend treten drei grundlegende Probleme bei einem Konfigurationsprozess auf:
\begin{itemize}

		\item zusätzlicher Kommunikationsaufwand durch zu spätes Feedback

        \item Daten werden doppelt erfasst.
        \item Produktnummern sind für den Kunden schwierig zu handhaben.
        
        
\end{itemize}
Diese drei Punkte müssen im neuen Workflow verbessert werden.

\subsection{Übertragung auf den Kunden}
Beim Anwendungsbeispiel des Kunden erfolgt die Auswahl der Upgrades ebenfalls über einen Produktkatalog. In diesem Katalog sind die einzelnen Upgrades aufgeführt. Die Auswahl der Produkte erfolgt über eine vorhandene Weboberfläche. Diese Oberfläche ist über die Homepage des Kunden verfügbar. Der Endkunde, im Anwendungsbeispiel eine Fluggesellschaft, wählt das gewünschte Upgrade aus dem Katalog aus. Bei der Bestellung werden die Produktcodes der Auswahl verwendet. Zusätzlich müssen die Flugzeuge angegeben werden, die das Upgrade erhalten sollen. Die Identifizierung erfolgt ebenfalls anhand des eindeutigen Flugzeugcodes. Beide Codes werden im nächsten Schritt von einem Produktkonfigurator erfasst und anschließend eine Überprüfung der Konfiguration durchgeführt.   \par 

Beim Kunden wird die klare Trennung der Auswahl und der Überprüfung durch die Verwendung von zwei unterschiedlichen Systemen deutlich. Die Kommunikation der beiden Systeme erfolgt über die Produkt-, bzw. Flugzeugcodes. Bei der Überprüfung der Konfiguration ist der Kunde  nicht involviert. Der Experte bearbeitet die Bestellung und pflegt diese in das System ein. Bei der Umstellung des Prozesses muss auch dieser Vorgang der Konfiguration verstanden werden, um die Eignung für den neuen Workflow zu überprüfen. \par 
\begin{figure} [H]
\centering
\includegraphics[width=300px]{images/workflow_webgui}
\caption{Programmablauf des Anwendungsbeispiels}
\label{webguiAblauf}
\end{figure}

Abbildung \ref{webguiAblauf} zeigt den Ablauf einer Konfiguration mit dem derzeitigen System.
Im ersten Schritt wird das passende Flugzeugprogramm ausgewählt. Ein Programm ist eine grobe Einteilung für Flugzeuge nach deren Größe und Art. Die Auswahl ist eine erste Filterung der Datensätze. Des Weiteren werden mit Hilfe des Programms die Regeln ausgewählt, die auf dem Konfigurationsserver verwendet werden. Anschließend folgt die Auswahl der entsprechenden Flugzeuge, welche ein Upgrade erhalten sollen. Bei der Identifizierung für eine Bestellung wird die Flugzeugnummer angegeben. Sind die Flugzeuge ausgewählt, werden Upgrades aus einer Liste selektiert. Die Auswahl erfolgt ebenfalls mit der eindeutigen Nummer, welche bei der Bestellung angegeben wird. \par 

Nach dem vorigen Schritt sind alle für die Konfiguration benötigten Elemente ausgewählt. Es folgt eine Validierung der Flugzeuge. Bei dieser Überprüfung werden die einzelnen Flugzeuge auf Konfigurationen untersucht, die in Widerspruch mit dem ausgewählten Upgrade stehen. Wenn keine Widersprüche vorhanden sind, werden sogenannte Konfigurationsgruppen gebildet. Eine Konfigurationsgruppe enthält Flugzeuge, die in die gleichen Zielzustände kommen, wenn das Upgrade eingebaut wird. Wenn es mehrere Möglichkeiten gibt, um in einen bestimmten Zustand des Flugzeuges zu kommen, sind sogenannte Alternativen in einer Konfigurationsgruppe enthalten. Damit die Konfiguration vollständig ist, muss der Anwender für die Gruppe eine Alternative auswählen. \par

Nachdem eine vollständige Konfiguration erzeugt wurde, wird daraus ein Excel-Dokument generiert. In diesem sind die Upgrades enthalten, die in den einzelnen Flugzeugen eingebaut werden müssen. Aus dem Dokument wird ein Upgrade-Angebot erstellt, das anschließend dem Kunden vorgelegt wird. \par 

Das Hauptproblem des aktuellen Systems ist, dass nur Experten die Anwendung bedienen können. Eine effektive Nutzung kann nur mithilfe der Produktcodes erfolgen. Dies führt dazu, dass man ein großes Wissen über die Produktstruktur besitzen muss. Dadurch kann der Kunde die Konfiguration mit der aktuellen Anwendung nicht selbstständig durchführen. Dieses Problem wird im Folgenden bei der Modellierung des neuen Workflows gelöst.

\section{Workflow Modellierung}\label{workflow_modelling}
Der neue Workflow muss die im vorigen Abschnitt erwähnten Probleme des Konfigurationsprozesses lösen.  Anschließend müssen die Lösungen auf den konkreten Anwendungsfall übertragen werden und ein neuer Programmablauf gefunden werden.

\subsection{Mobiler Konfigurationsprozess}\label{mobileConfiguration}
Die Probleme mit der Verwendung von Produktnummern und die doppelte Erfassung der Daten lassen sich durch die Zusammenlegung von Auswahl und Konfigurationsprüfung mit einem System lösen. Dieses Zusammenlegen von Produktauswahl und Produktkonfiguration in einer Anwendung ermöglicht es dem Kunden die gewünschte Auswahl zu tätigen und gleichzeitig ein Feedback der Konfiguration zu erhalten. Durch die bessere Rückmeldung verringert sich der Kommunikationsaufwand, da der Kunde im Idealfalls sofort das Resultat sieht. \par 
 Für eine noch bessere Unterstützung, sowie Hilfestellung beim Aufkommen von Alternativen ist die Durchführung des Prozesses mit einem Mitarbeiter des Herstellers von Vorteil. Dieser kann den Kunden durch die Konfiguration führen oder dabei unterstützen. Gleichzeitig wird hier ein besseres Verständnis für das Produkt ermöglicht. Der Hersteller hat den Vorteil einer Beschleunigung des Prozesses von der Produktauswahl bis zur Angebotserstellung. Er kann durch die direkte Auswahl der Produkte, sowie ein sofortiges Prüfen und ggf. eine Selektion der Alternativen die Konfiguration abschließen und ein  Angebot erstellen. \par 
 
Diese Umstellungen des Prozesses setzt die Verwendung eines mobilen Endgerätes, in diesem Fall eines Tablet-PCs voraus. Durch die Mobilität des Gerätes kann die Konfiguration direkt beim Kunden vor Ort durchgeführt werden. Die verbesserte Kommunikationsmöglichkeit mithilfe des Tablets sorgt für ein besseres Verständnis des Kunden und dessen Wünsche. 


Zusammenfassend sind folgende zwei Maßnahmen beim neuen Workflow durchzuführen: \par

\begin{itemize}
        \item Zusammenlegen von Produktauswahl und Produktkonfiguration. 
        \item Unterstützung durch einen Mitarbeiter des Herstellers.
\end{itemize}
Die konkrete Umsetzung dieser beiden Änderungen wird im Folgenden am Anwendungsbeispiel durchgeführt. \par

\subsection{Workflow des Anwendungsbeispiels}\label{workflowNew}
Für die Umsetzung der aus Abschnitt \ref{mobileConfiguration} erstellten Maßnahmen müssen die beiden vorhandenen Systeme für die Auswahl und Konfiguration auf ein gemeinsames Gerät portiert werden. Das Konfigurationssystem muss hierfür vereinfacht werden, um den Zugang für den Kunden zu erleichtern. Die Herausforderung besteht hier in einer übersichtlichen Darstellung der Konfigurationsergebnisse. Diese müssen für den Kunden nachvollziehbar aufbereitet werden, da er nicht die Produktkenntnisse des Experten besitzt. Da der Kunde die Auswahl der einzelnen Produkte "'live"' vornimmt, muss die Anwendung ein schnelleres Feedback erzeugen. Bei der vorherigen Lösung hat der Experte die komplette Zusammenstellung des Kunden erhalten und musste diese in das System übertragen. Beim neuen Workflow dagegen möchte der Kunde die Reihenfolge bei der Auswahl selbst bestimmen. Dies muss im neuen Anwendungsverlauf berücksichtigt werden. \par 
\begin{figure}
\centering
\includegraphics[width=\hsize]{images/workflow_app}
\caption{Workflow der Konfigurator-App}
\label{appWorkflow}
\end{figure}
In Abbildung \ref{appWorkflow} ist der Anwendungsverlauf der App zu sehen. Analog zu der Weboberfläche wird bei einer neuen Konfiguration zuerst ein Programm (\textbf{Programm-auswahl}) ausgewählt. Dies ist aufgrund einer Filterung der Daten weiterhin notwendig. Nach der Auswahl gelangt der Benutzer in eine \textbf{Zusammenfassung} der aktuellen Konfiguration. Von dieser Ansicht aus können Flugzeuge (\textbf{Flugzeugauswahl}) oder  Upgrades (\textbf{Upgradeauswahl}) selektiert werden. Hier wird den unterschiedlichen Bedürfnissen des Kunden entsprochen und kein strikter Konfigurationsablauf vorgegeben, wie es im vorherigen System war. Mit dieser Umstellung wird dem Kunden die Möglichkeit gegeben die einzelnen Upgrades nacheinander zu wählen und jederzeit eine schnelle Änderung zu ermöglichen. Sobald mindestens ein Flugzeug oder ein Upgrade ausgewählt ist, wird die Konfiguration überprüft. 
Nach der Überprüfung werden die  Konfigurationsgruppen, die der Konfigurator erstellt hat angezeigt. Diese Gruppen kann der Benutzer einsehen und bei mehreren Alternativen in einer 
separaten Ansicht (\textbf{Alternativenauswahl}) die richtige Lösung auswählen. Ist die Konfiguration vollständig, so hat der Nutzer die Möglichkeit die aktuelle Zusammenstellung zu bestellen und den Vorgang mit der Bestellung zu beenden. 

Da die Verwendung des Konfigurationsservers, wie in Kapitel \ref{configurator} beschrieben nach dem Client-Server Modell aufgebaut ist, wird eine andere Möglichkeit bei der Konfiguration nötig. Beim mobilen Einsatz der Anwendung kann es passieren, dass eine Verbindung mit dem Konfigurationsserver nicht möglich ist. Aus diesem Grund muss es einen alternativen Weg des Workflows geben. Wenn der Konfigurationsserver nicht verfügbar ist, wird die Konfiguration gespeichert, damit eine Prüfung später durchgeführt werden kann. In diesem Fall ist der Prozesses mit der Speicherung der Konfiguration beendet.


\section{Anforderungsanalyse} \label{requirements}
Nachdem der grundlegende Prozess auf die Bedürfnisse des Kunden, sowie an die mobile Umgebung angepasst ist, können daraus die Anforderungen der Anwendung für das Anwendungsbeispiel festgelegt werden. Bei der Festlegung wird zuerst die allgemeine Anforderung erläutert, sowie anschließend auf das Anwendungsbeispiel spezifiziert. Für eine einfachere Darstellung wird im Folgenden zwischen Funktionalen und Nicht-Funktionalen Anforderungen unterschieden. 

\subsection{Nicht-Funktionale Anforderungen}\label{non_functional_requirements}
Die Nicht-Funktionalen Anforderungen betreffen alle Maßnahmen zur Vereinfachung, bzw. Unterstützung des Prozesses, sowie Anforderungen aufgrund des vorhandenen Systems. Diese speziellen Voraussetzungen sind für den Benutzer meistens nicht sichtbar und laufen im Hintergrund, bzw. unbewusst ab. Aus diesem Grund müssen hier auch spezielle Gütekriterien festgelegt werden, um am Ende eine Evaluation zu ermöglichen. Als Basis für diese Anforderungen werden die 10 Heuristiken für Benutzerschnittstellen von Nielsen \cite{bib:heuristicsNielsen} verwendet.
Es lassen sich aus dem neuen Workflow folgende Anforderungen spezifizieren:


\paragraph{Einfache Bedienung:} Dadurch, dass der Kunde kein Experte ist und die Anwendung nicht jeden Tag verwendet, muss die Bedienung für den Kunden einfach sein. Je weniger bei der Verwendung der Software erklärt werden muss, desto besser ist diese Anforderung erfüllt. Für die Erfüllung dieses Ziels sollen folgende Schwerpunkte der Softwareergonomie \cite{bib:softwareErgonomie} behandelt werden: 
\begin{itemize}
        \item Selbstbeschreibungsfähigkeit
        \item Lernförderlichkeit
        \item Erwartungskonformität
\end{itemize}
Nielsen verwendet hierfür die Begriffe Konsistenz, Erkennung vor Erinnerung und Hilfe bzw. Dokumentation.

\paragraph{Schnelle Bedienung:} Die zweite Zielgruppe der Anwendung ist der Experte. Für diesen muss es die Möglichkeit einer schnellen Bedienung der Anwendung geben, sodass er möglichst effizient die Konfiguration gestalten kann. Somit wird eine flexible Bedienung der Anwendung nötig. Für eine zusätzliche Steigerung der Effizienz muss die Navigation zwischen den einzelnen Seiten schnell sein. Dies bedeutet, dass beim Übergang von einer zur nächsten Seite der Benutzer nicht lange warten sollte. Ebenfalls muss es genügend Feedback geben, wenn ein Seitenwechsel länger dauern sollte. Hier werden die beiden Heuristiken Sichtbarkeit des aktuellen Status und Flexibilität, sowie Effizienz bei der Benutzung beachtet.

\paragraph{Optimierung auf die Umgebung: } Da ein mobiles Gerät nur begrenzte oder eingeschränkte Resourcen zur Verfügung hat, müssen die einzelnen Bedienelemente auf die Umgebung angepasst sein. Ebenfalls muss die Bedienung für eine Touch-Eingabe optimiert sein. 
Für ein einheitliches Aussehen der Anwendung müssen die Richtlinien der jeweiligen Technologie beachtet werden. Diese Anforderung enthält die beiden Heuristiken ästhetisches und minimalistisches Design, sowie Standards und Konsistenz.

Für eine bessere Übersicht der Nicht-Funktionalen Anforderungen sind diese in der Tabelle \ref{nonFunctionalRequ} dargestellt. Zu jeder Anforderung wird hier ein zusätzlicher Bezeichner definiert.


\begin{tabular}[H]{| p{0.7cm} | p{2.2cm} | p{4.5cm} | p{5.5cm}|}

\toprule[2pt] \rowcolor{dunkelgrau}
\hline
  Bez. & Anforderung & Beschreibung & Heuristiken nach Nielson \\
  \hline
  N1 & Einfache \newline Bedienung & Die Anwendung soll von unerfahrenen Benutzern bedient werden können.& \begin{itemize}
          \item Sichtbarkeit des aktuellen Status
          \item Erkennung vor Erinnerung
          \item Hilfe und Dokumentation
  \end{itemize} \\
  \hline
  N2 & Schnelle \newline Bedienung & Experten müssen die Anwendung schnell und effizient bedienen können. & \begin{itemize}
            \item Flexibilität und Effizienz bei Benutzung
            \item Sichtbarkeit des aktuellen Status
    \end{itemize}  \\
  \hline
    N3 & Optimierung auf Umgebung & Die Anwendung muss auf die Zielplattform optimiert werden. &  \begin{itemize}
              \item Ästhetisches und minimalistisches Design
              \item Standards und Konsistenz
      \end{itemize} \\
    \hline
\bottomrule[2pt]
\end{tabular}
\label{nonFunctionalRequ}

\subsection{Funktionale Anforderungen}
Die Funktionalen Anforderungen sind von der Prozessbeschreibung im Abschnitt \ref{mobileConfiguration} abhängig. Im Folgenden werden diese speziell auf das Anwendungsprojekt spezifiziert. Zusätzlich zu der Spezifikation wird der Bezug zum neuen Workflow hergestellt:

\paragraph{Filterung der Anwendungsdaten: } Da sehr viele Daten vorhanden sind, müssen diese für eine übersichtliche Darstellung im Vorfeld gefiltert werden. Die Filterung erfolgt in zwei Schritten:
\begin{enumerate}
\item Auswahl der Kundendaten: Die Anwendung soll speziell bei einem Kunden eingesetzt werden. Aus diesem Grund kommen nur die kundenspezifischen Daten zum Einsatz.
\item Programmauswahl: Durch die Wahl des Programmes wird eine Filterung der einzelnen Flugzeuge und der möglichen Updates erreicht.
\end{enumerate}
Diese Anforderung wird ebenfalls durch die Verwendung einer mobilen Umgebung notwendig. Damit der zweite modellierte Workflow mit einer nicht vorhandenen Internetverbindung funktioniert, müssen die Daten offline verfügbar sein. Da nur begrenzte Speichermöglichkeiten auf dem Gerät vorhanden sind, können nicht alle Kundendaten gespeichert werden. Dies bedingt eine Filterung durch die Auswahl des Kunden.

\paragraph{Upgradeauswahl: } Bei der Auswahl von Upgrades im Anwendungsbeispiel werden die Funktionen des Produktkataloges benötigt. Es muss möglich sein, ein bestimmtes Produkt an- oder abzuwählen. Damit der Kunde den Aufbau des Produktes versteht, müssen die einzelnen Produkte nach einer bestimmten Struktur geordnet sein. 

\paragraph{Flugzeugauswahl: } Flugzeuge, die ein ausgewähltes Upgrade erhalten sollen, müssen ebenfalls in der Anwendung auswählbar sein. Die Anforderungen an diese Auswahl sind nicht die Gleichen wie bei der Produktauswahl. Der Schwerpunkt muss hier auf ein schnelles Finden der einzelnen Flugzeuge gelegt werden. Dies wird ebenfalls durch eine dem Kunden bekannte Struktur erreicht. 

\paragraph{Anzeige der Konfigurationsergebnisse: } Damit der Kunde ein schnelles Feedback der aktuellen Auswahl erhält, müssen die Konfigurationsergebnisse angezeigt werden. Dies impliziert eine Anbindung an den Konfigurationsserver, der die entsprechenden Ergebnisse berechnet. Diese Anzeige muss eine vereinfachte Darstellung beinhalten, damit der Kunde diese Ergebnisse verstehen kann.

\paragraph{Alternativenauswahl: } Damit eine vollständige Konfiguration erzeugt werden kann, muss die Anwendung eine Auswahl für Alternativen bereitstellen. Hierbei muss das Verständnis des Kunden für das Produkt berücksichtigt werden. Die Alternativen müssen auf eine verständliche Art dargestellt sein. 

\paragraph{Speichern und Laden der Konfiguration: } Die zweite Voraussetzung des alternativen offline Workflows ist eine Speicherung der Konfiguration. Für eine spätere Bearbeitung ist das erneute Laden der Konfiguration notwendig. Diese beide Anforderungen werden aufgrund der mobilen Zielumgebung benötigt.

Die Zuordnung der einzelnen Anforderungen zu den jeweiligen Prozessschritten ist in Tabelle \ref{functionalRequirements} zu sehen. Hierbei werden die drei Prozessanforderungen Produktkatalog, Konfiguration und mobile Zielumgebung unterschieden.
\par 
\begin{tabular}[H]{| p{0.4cm} | p{2.5cm} | p{5.9cm} | p{4.5cm} |}
\toprule[2pt] \rowcolor{dunkelgrau}
\hline
  NR. & Anforderung & Beschreibung & Prozesszuordnung \\
  \hline
  F1 & Upgrade-auswahl & Es sollen Upgrades für Flugzeuge auswählbar sein & Produktkatalog
   \\
  \hline
  F2 & Flugzeug-auswahl & Es müssen Flugzeuge eines bestimmten Kunden auswählbar sein. & Produktkatalog  \\
  \hline
    F3 & Konfigurations-ergebnisse anzeigen & Übersichtliche Darstellung der aktuellen Zusammenstellung & Konfiguration \\
    \hline
     F4 & Alternativen-auswahl & Bei mehreren Möglichkeiten einer Auswahl sollen Alternativen ausgewählt werden können & Konfiguration \\
        \hline
    F5 & Speichern und Laden & Die getätigte Auswahl soll gespeichert und geladen werden können& mobile Zielumgebung \\
    \hline
    F6 & Filterung der Anwendungsdaten & Die Daten (Flugzeuge und Upgrades) müssen gefiltert werden können & mobile Zielumgebung \\
    \hline
\bottomrule[2pt]
\end{tabular}
\label{functionalRequirements}






	\chapter{Entwurf der Benutzerschnittstelle}\label{chapter_4}
Die Anforderungen der Anwendung für die Unterstützung des neuen Workflows wurden im vorigen Kapitel abgeleitet. Diese Kriterien werden im nächsten Schritt in einem Entwurf der Benutzerschnittstelle umgesetzt. Bevor die Schnittstelle entworfen werden kann, muss die geeignete Anwendungsform und eine darauf basierende Technologie ausgewählt werden. Anschließend folgt eine kurze Analyse der Zielplattform und deren Konzepte, damit diese beim Entwurf berücksichtigt werden können, bevor die einzelnen Ansichten entworfen werden.

\section{Auswahl der Anwendungsplattform}
Aufgrund der vielen Möglichkeiten bei der Entwicklung von mobilen Anwendungen muss die Art und Technologie der Anwendung richtig gewählt werden. Hierbei soll zuerst die passende Anwendungsform (Nativ, Web oder Hybrid siehe \ref{mobileAppsGrundlagen}) gewählt werden. Durch diese Vorauswahl, wird die Anzahl der möglichen Technologien begrenzt, was im folgenden Schritt die Auswahl vereinfacht.


\subsection{Anwendungsform}
Damit die richtige Anwendungsform der App ausgewählt werden kann, müssen die drei Möglichkeiten Nativ, Web und Hybrid auf ihre Eignung bei der Umsetzung der Anforderungen untersucht werden. Wichtig bei der Entscheidung ist die Festlegung der konkreten Kriterien. 
Die Umsetzung der rein fachlichen Funktionen Produktauswahl und Konfiguration kann mit jeder Anwendungsform durchgeführt werden. Hier sind die Unterschiede für eine Auswahl nicht ausreichend. \par
Bei den  Funktionalen Anforderungen sind die zwei Kriterien, die aufgrund der mobilen Zielumgebung wichtig sind für die Auswahl bedeutend. Diese sind das Speichern und Laden (F5), sowie die Filterung der Anwendungsdaten (F6). Für die Umsetzung dieser Funktionen ist eine Verwendung des Dateisystems auf der mobilen Zielumgebung notwendig. Hier muss eine Form der Speicherung, ob Datenbank oder einfaches Speichern in einer Datei möglich sein. Diese Voraussetzung ist für die Entscheidung der Anwendungsform essentiell. \par

Das zweite Kriterium leitet sich von den Nicht-Funktionalen Anforderungen einer schnellen Bedienung (N2) ab. Damit die Anwendung schnell bedient werden kann, müssen die vorhandenen Hardwareressourcen optimal verwendet werden können. Dies ist notwendig, um bspw. das Laden von Bildern zu beschleunigen. Ebenfalls müssen Seitenübergänge ohne große Wartezeiten möglich sein, um die Anforderung erfüllen zu können. Die Anwendungsform muss für die Umsetzung Schnittstellen bereitstellen, die auf die vorhandene Hardware optimiert sind.

Für eine Optimierung der Anwendung auf die Zielumgebung (N3) ist ein ästhetisches und minimalistisches Design die entscheidende Heuristik für die Auswahl der Anwendungsform. Für die Implementierung muss eine gute optische Integration in das System vorhanden sein. Dies ist für eine übersichtliche Aufbereitung des Produktkataloges eine Voraussetzung. Der App-Typ muss Oberflächenelemente zur Verfügung stellen, die zu der Gesamtumgebung passen. Durch eine gute Integration in die Zielumgebung wird dadurch die Anwendung ästhetischer. Die gesamte Wahrnehmung der App wird hierdurch verbessert. Das dritte Zielkriterium für die Auswahl ist somit die Verwendung von betriebssystemspezifischen Oberflächenelementen.    

Im Folgenden werden die drei festgestellten Kriterien lokale Speicherung der Daten, Hardwarenahe Schnittstellen und Verwendung von betriebssystemspezifischen Oberflächenelementen  für jede Anwendungsform bewertet, sodass am Ende eine Gegenüberstellung stattfinden kann.

\paragraph{Native Anwendung: }Bei Nativen Anwendungen ist der komplette Funktionsumfang der mobilen Zielplattform verfügbar. Dies ermöglicht ein breites Anwendungsfeld für native Anwendungen.
\begin{itemize}
\item \textbf{Lokale Speicherung der Daten:} Die native Anwendungsform stellt Schnittstellen für den Zugriff auf eine lokale Datenbank oder ein lokales Dateisystem bereit. Diese können ohne Anpassungen verwendet werden. 

\item \textbf{Hardwarenahe Schnittstellen:} Dadurch, dass die einzelnen Schnittstellen direkt vom Hersteller der jeweiligen Plattform kommen, sind Operationen für das System optimiert. Mit einer nativen Anwendung wird damit die beste Performanz erreicht. 

\item \textbf{Betriebssystemspezfische Oberflächenelemente:} Native Anwendungen verwenden für die Benutzerschnittstelle die spezifische Oberflächenelemente. Die Steuerung mit Touch ist ebenfalls auf diese Anwendungsform optimiert. Dies ergibt eine vollständige Verwendungsmöglichkeit des bereitgestellten Frameworks.
\end{itemize}

Auf die einzelnen Kriterien bezogen erfüllt die native Anwendung alle Anforderungen. Der volle Funktionsumfang des Systems ist gegeben, wodurch die App ideal für die gewählte Zielplattform geeignet ist.

\paragraph{Web Anwendung: } Web Anwendungen werden im Browser ausgeführt. 
 Dies führt dazu, dass nur Funktionen verwendet werden können, die der jeweilige Browser auf der Zielplattform zur Verfügung stellt. 
 \begin{itemize}
 \item \textbf{Lokale Speicherung der Daten:} Ein Speichern und Laden der Daten vom Betriebssystem ist bei einer Web Anwendung nicht ohne weiteres möglich. Der Browser verbietet durch das Sandbox-Modell einen direkten Zugriff auf das System.  Ein Ansatz zur Lösung des Problems ist die Installation des Servers direkt auf dem Zielsystem, so dass dieser lokal zur Verfügung steht. Eine Erfüllung der Anforderungen ist damit möglich, jedoch mit erheblichem Mehraufwand. 
 
 \item \textbf{Hardwarenahe Schnittstellen:} Die Kommunikation mit dem Betriebssystem erfolgt durch den Browser. Diese zusätzliche Zwischenschicht sorgt dafür, dass die Hardware nicht direkt verwendet wird, wie es bei einer nativen Anwendung der Fall ist. Der zusätzliche Overhead sorgt dafür, dass die Performance bei einer Web Anwendung nicht ideal im Vergleich zur nativen Lösung ist.
 
 \item \textbf{Betriebssystemspezfische Oberflächenelemente:} Die grundlegenden Oberflächenelemente sind HTML Elemente. Diese können mit Javascript und CSS einen passenden Stil für das jeweilige Betriebssystem erhalten. Die Verwendung von speziellen Touch Gesten ist mit Web Anwendungen aufgrund der Interoperabilität mit mehreren Betriebssystemen schwieriger. Hier sind ebenfalls nur begrenzte Interaktionen möglich.
 
 \end{itemize}
 Der Hauptvorteil der Web Anwendung die Verwendung, ohne komplette neue Implementierung auf mehreren Systemen ist aufgrund der Kriterien nicht relevant. Die wichtigen Eigenschaften der Anwendung Performanz und lokale Speicherung können mit dieser Anwendungsform gelöst werden, jedoch deutlich schlechter als bei der nativen Variante.


\paragraph{Hybride Anwendung: }Durch die Verwendung der lokalen Schnittstellen im nativen Anwendungscontainer (siehe \ref{hybridApplication}) können die vorhanden Ressourcen wie bei einer Nativen Anwendung verwendet werden. Dies führt zu einigen Verbesserungen gegenüber einer reinen Web Anwendung.

\begin{itemize}
 \item \textbf{Lokale Speicherung der Daten:} Für die Umsetzung dieser Anforderung muss eine geeignete Schnittstelle im Anwendungscontainer zur Verfügung gestellt werden. Hierdurch wird eine problemlose Verwendung des Dateisystems möglich. Das Kriterium kann ohne Einschränkungen erfüllt werden. 
 
 \item \textbf{Hardwarenahe Schnittstellen:} Durch den Anwendungscontainer wird, wie bei einer Webanwendung der Browser, eine Zwischenschicht nötig. Damit wird ein zusätzlicher Kommunikationsaufwand benötigt. Dieser zusätzliche Aufwand verringert die Performanz der Anwendung.
 
 \item \textbf{Betriebssystemspezfische Oberflächenelemente:} Die Oberflächenelemente des Betriebssystems lassen sich im Container bereitstellen. Somit können die nativen Elemente in der Hybriden Anwendung verwendet werden. Die Verwendung der Touch Bedienung wird durch dieses Konzept ermöglicht.
 \end{itemize}
 Die Probleme, die bei der Verwendung einer Web Anwendung auftreten können durch den hybriden Ansatz gelöst werden. Ein Problem was weiterhin besteht, ist die Performanz. Diese wird mit der hybriden Anwendungsform besser, kommt jedoch nicht an die Leistung einer nativen Anwendung heran.

\paragraph{Abwägung: }
Nachdem alle Anwendungsformen auf die Erfüllung der aufgestellten Kriterien untersucht sind, kann eine Gegenüberstellung der einzelnen Komponenten erfolgen.  Bei der ersten Betrachtung der Anwendungsformen ist die Implementierung als Web Anwendung nicht sinnvoll. Die Anforderungen können zwar umgesetzt werden, jedoch ist ein erheblicher Mehraufwand für die Erreichung nötig. Ebenfalls hat die Anwendung im Vorhinein bereits Einschränkungen bzgl. der Performanz, sowie der Bereitstellung von Oberflächenelementen. Die Vorteile einer Implementierung für mehrere Zielsysteme ist bei der Umsetzung kein Kriterium. \par

Schwieriger ist die Entscheidung zwischen hybrid und nativ. Die beiden Kriterien lokale Speicherung der Daten und betriebssystemspezifsche Oberflächenelemente sind mit beiden Ansätzen ohne größere Einschränkungen möglich. Beim hybriden Ansatz ist ein geringer Mehraufwand bei der Implementierung nötig. Dies ist jedoch nicht relevant für die Entscheidung. Der wichtigste Grund für die Entscheidung für eine nativen Anwendung ist die Performanz bei der Bedienung. Nur durch die Verwendung von nativen Anwendungen können aufwändige Animationen oder das Rendern von vielen Bildern flüssig ablaufen. Eine hybride Lösung besitzt hier Einschränkungen, da die Hardware mit einer Zwischenschicht verwendet wird.

\subsection{Anwendungstechnologie}
Mit der Entscheidung für eine mobile Anwendungsform kommen drei mögliche Technologien für die Implementierung in Frage. Die Plattformen Android \footnote{http://www.android.com/}, iOS \footnote{http://www.apple.com/de/iphone/ios/} und Windows 8 \footnote{http://windows.microsoft.com/de-de/windows-8/} sind aufgrund ihrer Marktanteile (Android: 43,4\% iOS: 48,2\% Windows: 7,4\% Quelle: \cite{bib:marktanteilBS} ) die wichtigsten Technologien im Tablet Bereich. Die drei Plattformen haben unterschiedliche Ziele. Das iOS Betriebssystem hat den Vorteil einer hohen Verbreitung bei Business Anwendungen (siehe \cite[S.5]{bib:mobileMarketing2}). Android hat die meisten Nutzer bei Privatanwendungen (siehe \cite{bib:marktanteilMBS}). Bei Windows 8 ist der Vorteil eines neuen Konzeptes, was speziell für Tablet-PCs optimiert ist. Für die Umsetzung des Workflows kann jede dieser Technologien verwendet werden. Ist die Implementierung mit einem Framework umgesetzt, stellt das Implementieren auf einer anderen Zielplattform keine Herausforderung da. \par 

Bei der vorliegenden Arbeit wird auf Windows 8 gesetzt. Im Gegensatz zu iOS und Android wird dieses Betriebssystem nicht auf Smartphones ausgeführt. Dies führt zu einer besseren Optimierung für größere Bildschirme. Die Technologie ist sehr neu auf dem Markt (26. Oktober 2012), dadurch gibt es noch nicht viele Apps. Dies bietet die Möglichkeit den Anwender mit neuen Möglichkeiten in der Anwendung zu überraschen, da einige Konzepte noch nicht bekannt sind. Die Möglichkeit zu experimentieren und neue Ideen umzusetzen ist aufgrund der neuen Plattform vorhanden.  Die Verwendung von Windows 8 vereinfacht ebenfalls die Integration in ein Unternehmensumfeld, da hier die Microsoft Produkte weit verbreitet sind. Dies führt zu einer schnelleren Akzeptanz im Unternehmen. Somit wird im Folgenden die Anwendung für das Windows 8 Betriebssystem implementiert.
        

\section{Untersuchung der Plattform Windows 8}
Damit die Anforderung eines ästhetischen und minimalistischen Designs (N3) erfüllt ist, muss vor der Gestaltung der einzelnen Ansichten die Zielplattform untersucht werden. Die Designgrundlagen müssen verstanden werden, um ein passendes Aussehen realisieren zu können. 
Die Folgende Untersuchung wird in drei Teile aufgeteilt. Im ersten Abschnitt werden allgemeine Design Prinzipien behandelt, darauf aufbauend die Bedienkonzepte und touchoptimierten Bedienelemente der Plattform.

\subsection{Design Richtlinien}
Damit einheitliche Apps entwickelt werden, hat Microsoft Richtlinien (Entnommen aus: \cite{bib:win80}, \cite{bib:win81}, \cite{bib:win82}, \cite{bib:win83}) aufgestellt.
Das wichtigste Design Element bei einer Windows 8 Anwendung sind sogenannte Kacheln. Diese Kacheln sind meist quadratisch und in jeder App vorhanden . Jede Funktion wird über eine Kachel erreicht. Sie soll mehr Informationen darstellen, als ein einfaches Logo oder Icon auf einem Button. Durch einen dynamischen Inhalt und unterschiedliche Größen wird dadurch dem Benutzer ein neues Benutzererlebnis gegeben. Die Organisation der Kacheln erfolgt in einem sogenannten Grid (engl. für Gitter). Dieses besteht aus mehreren Quadraten. Das kleinste Quadrat hat eine Größe von einem Pixel. Das Grid besitzt verschiedene Bereiche für die Überschrift und den Inhalt. Dieser ist vorgegeben und sollte eingehalten werden. \par 

Die Anwendung muss sich auf die Anzeige des Wesentlichen konzentrieren. Microsoft nennt  das Prinzip "'Content over Chrome"'.  Für die Umsetzung dieser Richtlinie soll die Anwendung nur die wichtigsten Funktionen in der Ansicht darstellen. Es sollen überladene Ansichten vermieden werden und stattdessen bewusst größere Elemente mit mehr Platz verwendet werden. 
 
Ein weiteres wichtiges Design Element ist die Typographie. Hier geht es um die bewusste Gestaltung von Schriften. Die Kalligrafie wird als Vorbild verwendet.  Die Idee ist keine rein statische Verwendung der Texte. Der Nutzer soll die Möglichkeit haben diese auszuwählen, damit eine Interaktion ermöglicht wird. 

Diese Richtlinien werden beim Entwurf der einzelnen Ansichten berücksichtigt. 


\subsection{Bedienkonzepte}
Bei den Konzepten für die Bedienung steht bei Windows 8 eine besondere Optimierung für Tablet-PCs im Vordergrund. Es werden deshalb sehr viele Gesten verwendet. Eine wichtige Geste ist das sogenannte "'wischen"'. Diese Aktion ist von jeder Seite des Bildschirms erlaubt. Beim Wischen von oben oder unten wird die sogenannte AppBar eingeblendet. Diese Bar wird an der oberen Seite für die Navigation durch die Anwendung verwendet. Dies ermöglicht einen schnellen Wechsel zwischen den Ansichten. Die untere AppBar wird für Aktionen verwendet, die nicht Vordergrund stehen. Ein Beispiel wäre die Filterung der Eingabedaten nach bestimmten Kriterien. Ein Wischen von der rechten Seit lässt die sogenannte CharmBar erscheinen. Der Inhalt dieser Bar ist die Verwendung von sogenannten Contracts (engl. Verträge). Diese Funktionen werden für alle Anwendungen durch das Betriebssystem bereitgestellt. An dieser Stelle können Einstellungen oder Suchen durchgeführt werden. \par 

Das zweite wichtige Bedienkonzept ist das horizontale Scrollen. Aufgrund der größeren Bedienelemente sind nicht immer alle Elemente sichtbar. Die Lösung ist ein horizontales Ausbreiten des Inhalts. Damit die Inhalte verwendet werden können, wird ein horizontales Scrollen mit einer Wisch-Geste durchgeführt. Hier muss darauf geachtet werden, dass ein Ausschnitt des nächsten Elementes sichtbar ist, damit dem Benutzer eine Erweiterung der Ansicht signalisiert wird. \par 

Beim Aufbau der Anwendung kann entweder eine hierarchische oder  flache Struktur verwendet werden. Für den in Abschnitt \ref{workflowNew} erstellen Workflow ist eine hierarchische Architektur passender. Der Ursprung geht hier immer von der Startseite, der sogenannten Hub-Page aus. Diese Seite ist der zentrale Startpunkt von der alle weiteren Aktionen ausgehen. Die zweite Ebene sind sogenannte Section Pages. Diese stellen den Inhalt einer  Kategorie dar. Die unterste Ebene sind die Detail Pages. Diese enthalten die jeweiligen Details eines Elementes in einer Kategorie. Bei einer Zeitungs App wäre beispielsweise auf der Startseite die einzelnen Kategorien wie Politik, Wirtschaft oder Sport zu sehen. In der Kategorie Ansicht die jeweiligen Überschriften der Artikel. Die Detail Seite würde den Artikel zeigen. Dieser baumartige Aufbau wird durch das Verwenden eines Zurück Buttons unterstützt, der in jeder Ansicht, außer der Startseite vorhanden ist. Durch diesen Button wird dem Benutzer eine weitere Navigationsmöglichkeit gegeben.

Beim Design der Ansichten ist ein durchdachtes Bedienkonzept aufgrund der beiden Anforderungen N1 und N2 wichtig. Die Möglichkeiten, die zur Verfügung gestellt werden, sollten beim ersten Entwurf der App enthalten sein.

\subsection{Touchoptimierte Bedienelemente}
Für die Unterstützung der Bedienung durch Gesten enthält das Framework besondere Oberflächenelemente. Diese sind für die Verwendung mittels Touch optimiert. Die in der Arbeit verwendeten Elemente werden im folgenden vorgestellt.
Die Design Richtlinien von Microsoft erzeugen Probleme bei der Darstellung von vielen Daten. Eine Umsetzung der Richtlinie "'Content over Chrome"', sowie die Darstellung auf einem Gerät mit kleinerem Bildschirm verursachen einen großen Aufwand beim Scrollen. Hier kann eine lange Zeit für das Auswählen eines bestimmten Datums benötigt werden. \par

Für die Lösung dieses Problems bietet Windows 8 den sogenannten Semantischen Zoom an. Diese Funktion wird mit einer Kneif-Geste auf dem aktuellen Datensatz durchgeführt. Hierdurch wird die Ansicht nicht optisch verkleinert. Es erfolgt ein Wechsel von der Detailansicht zu einer Kategorieansicht. Die Datensätze werden somit semantisch verkleinert, wodurch ein schnelles Navigieren zum gewünschten Datum möglich ist. Der Semantische Zoom darf nicht geschachtelt verwendet werden und ist damit auf eine Ebene beschränkt. Voraussetzung für die Verwendung des Zooms ist die Einteilung der Daten in Kategorien. Ohne diese Kategorien kann keine übergeordnete Ansicht erstellt werden.

Das zweite neue Bedienelement, welches für eine Touch-Optimierung dient, ist die sogenannte Flip-Ansicht. Mit ihr kann durch ein Wischen von der linken oder rechten Seite die Ansicht gewechselt werden. Dieses Element kann den Benutzer bei einem schnellen navigieren helfen. Es sollten jedoch nicht zu viele Elemente für den Wechsel vorhanden sein.

\section{Entwurf der Ansichten}
Die vorgestellten Konzepte der Anwendungsplattform, sowie die einzelnen Bedienelemente werden für den Entwurf der Ansichten verwendet. Der Workflow der App, wie in Abschnitt \ref{appWorkflow} beschrieben, wird hier als Grundlage verwendet. Alle Prozessschritte erhalten eine eigene Ansicht. Zusätzlich werden die Anforderungen aus Kapitel \ref{functionalRequirements} beim Entwurf berücksichtigt.
\subsection{Startseite}
\subsection{Produktkatalog}
\subsection{Flugzeugauswahl}
\subsection{Konfigurationsergebnisse}
\subsection{Navigationskonzept}

	\chapter{Implementierung}\label{chapter_5}

\section{Implementierung des Katalog-Workflows}
\section{Implementierung des Konfigurations-Workflows}

	\chapter{Evaluation der Anwendung}\label{chapter_6}
Die Anwendung wurde anhand der spezifizierten Anforderungen in Kapitel \ref{requirements} umgesetzt. Beim Entwurf der Ansichten sind die Heuristiken von Nielsen \cite{bib:heuristicsNielsen} sowie die Designrichtlinien von Microsoft für die Zielplattform verwendet worden. Eine Überprüfung, ob die Ziele, die zuvor gestellt wurden erreicht sind, wird mithilfe einer Evaluation der Anwendung sichergestellt. Der Aufbau und die Zielsetzung wird im Folgenden beschrieben. 

\section{Durchführung der Evaluation}
Der neue Workflow, wie in Kapitel \ref{chapter_3} beschrieben, ist auf die Verwendung mit einem mobilen Endgerät zugeschnitten. Aus diesem Grund ist eine Evaluation des Gesamtprozesses nicht zielführend. Wichtiger ist die Erfüllung der gestellten Anforderungen. Wenn diese erfüllt sind, so kann der modellierte Prozess umgesetzt werden. Für die Untersuchung der verschiedenen Anforderungstypen werden unterschiedliche Evaluationen durchgeführt.

\subsection{Funktionale Anforderungen}
Die funktionalen Anforderungen aus Abschnitt \ref{functionRequ} werden mit Anwendern evaluiert. Hierzu wird die alte Weboberfläche, die aktuell für die Konfiguration verwendet wird, als Ausgangspunkt verwendet. Da in der neuen Anwendung ein zusätzlicher Produktkatalog integriert ist, wird kein direkter Vergleich der beiden Lösungen durchgeführt. Stattdessen werden gezielte Aufgaben an die Benutzer gestellt, die eine Umsetzung der Anforderungen überprüft. Die einzelnen Aufgabenstellungen werden anhand einer Skala von 1-5 bewertet. Bei der Bewertung wird die Frage, wie gut oder schlecht die Aufgabe durchgeführt werden kann, für eine Bewertungsgrundlage verwendet. 

Da die aktuelle Konfigurationslösung für Experten entwickelt wurde, wird die Benutzerevaluation in einer Interview Form durchgeführt. Die auftretenden Fragen werden protokolliert, so dass eine zusätzliche Auswertung der Probleme im Anschluss erfolgen kann. Ebenfalls sind die Aufgaben so gestellt, dass sie mit beiden Anwendungen durchgeführt werden können. Da die Weboberfläche nicht die gleichen Funktionen besitzt, wie die App, werden die Anforderungen, die nur für die Tablet-Anwendung definiert wurden bei der Evaluation nicht berücksichtigt. 

\subsection{Nicht-Funktionale Anforderungen}
Eine Überprüfung der Nicht-Funktionalen Anforderungen ist eine größere Herausforderung, da hier meist subjektive Meinungen entstehen. Aus diesem Grund wurde bei der Spezifikation der Anforderungen die zehn Heuristiken von Nielson verwendet. Anhand dieser kann eine Auswertung der Anwendung erfolgen. Hierzu werden zwei unterschiedliche Arten von Tests durchgeführt. Die zuvor genannten Benutzertests werden mit zusätzlichen Fragen zur Verwendung der App erweitert. Die Fragestellungen beziehen sich auf eine subjektive Wahrnehmung der Nicht-Funktionalen Anforderungen. Für eine objektivere Sicht wird eine zusätzliche Expertenevaluation durchgeführt, bei der die zehn Heuristiken von Nielsen bewertet werden. Die Experten kennen sich mit dem System sowie mit Benutzerschnittstellen aus und können so gezielt die Eigenschaften des Systems untersuchen. Bei der Durchführung wird der Experte die Anwendung ohne Vorgaben überprüfen und anschließend eine Bewertung anhand der vorgegebenen Kriterien abgeben. 

\subsection{Auswahl der Testmenge}
Damit eine Evaluation sinnvoll ist, muss eine geeignete Anzahl von Testergebnissen vorliegen. Anhand der Untersuchungen von Nielsen \cite{bib:countTests} findet eine Anzahl von fünf Personen bereits 75\% aller Usability Probleme. Weiterhin hat eine Anzahl von drei Benutzern das beste Verhältnis von Kosten und Nutzen. Aus diesem Grund werden vier Benutzertests und drei Expertentests durchgeführt. 

\section{Ergebnisse}
Nach der Durchführung der Evaluation sind drei Ergebnisse vorhanden. Die jeweiligen Fragebögen der Experten und Benutzer sowie die Auswertung der Benutzerfragen. Bei der vorhandenen Konfigurationslösung sind Fragen zum Expertenwissen aufgetreten. Im Gegensatz dazu sind bei der Verwendung der Tablet Anwendung keine fachlichen Fragen aufgetreten. Dies zeigt die unterschiedlichen Zielgruppen der beiden Anwendungen. Eine wiederkehrende Frage ist beim Abschluss der Konfiguration aufgetreten. Hier ist nicht ersichtlich gewesen, wie diese Funktion durchgeführt werden kann. Diese Frage ist bei allen vier Testern aufgekommen. Die weiteren Aufgaben waren klar und konnten ohne Problem durchgeführt werden. 

Für die Auswertung der Fragebögen wurden die wichtigsten Stellen herausgezogen. Das vollständige Ergebnis der Evaluation sowie die einzelnen Bögen befinden sich im Anhang \ref{anhangEva}.


\subsection{Auswertung der Benutzerergebnisse}
\begin{figure}[H]
\centering
\includegraphics[width=\hsize]{images/bewertung_tablet}
\caption{Auswertung der Funktionalen Anforderungen im Fragebogen}
\label{bewertungTablet}
\end{figure}
Das Ergebnis der Fragebögen in Bezug auf die Funktionalen Anforderungen ist in Abbildung \ref{bewertungTablet} zu sehen. Auffällig ist die besonders gute Bewertung für die Auswahl einer neuen Konfiguration und der Alternativenauswahl. Diese beiden Ansichten und deren Funktionen konnten während des Tests schnell gefunden, verstanden und verwendet werden. Die drei Hauptfunktionen Flugzeugauswahl, Upgradeauswahl und Darstellung der Konfigurationsergebnisse sind durchschnittlich als Gut bewertet worden. Für eine optimale Verwendung hat hier das Wissen eines Kunden gefehlt, der seine eigenen Flugzeuge kennt sowie eine grobe Zuordnung der Upgrades in die einzelnen Bereiche vornehmen kann. \par

Das Problem mit dem Abschluss der Konfiguration, welches sich bereits bei den Fragen der Anwendern herausgestellt hat, ist auch in der Bewertung ersichtlich. Drei der vier Anwender haben dem Abschluss der Konfiguration mit befriedigend bewertet. An dieser Stelle sollte eine Verbesserung für die Erfüllung der Anforderungen stattfinden. \par 
\begin{figure}[H]
\centering
\includegraphics{images/bewertung_tabletComplete}
\caption{Ergebnisse der Usability Fragen zu der Tablet Anwendung}
\label{bewertungUx}
\end{figure}
Die Bewertungen des zweiten Teils der Evaluation sind in der Grafik \ref{bewertungUx} zu sehen. Besonders positiv fallen die Ergebnisse für die Bedienung mit dem Tablet sowie einer verständlichen Alternativenauswahl auf. Somit wurde ein Verständnis für die Auswahl von Alternativen geschaffen, was dabei hilft den Konfigurationsprozess mit dem Kunden zusammen durchzuführen. Diese Tatsache unterstützt die gute Bewertung beim Punkt selbständige Bedienung. Die Maßnahmen für ein flexibles und tolerantes Navigationskonzept sind im Kriterium übersichtliche Navigation sehr gut bewertet worden.

Das Ergebnis für eine gute Bedienbarkeit auf dem Tablet resultiert auf der ausgewählten Plattform sowie den gewählten Bedienelementen in der Entwurfsphase. Bei der Implementierung wurde der Entwurf sehr gut umgesetzt, wodurch die gute Bewertung zustande kommt. \par 

Beim ersten Verwendung der Anwendung reagiert die App langsamer als beim zweiten Mal. Der Grund hierfür ist, dass beim initialen Start die Daten erst in den Hauptspeicher geladen werden müssen. Bei der nächsten Verwendung sind diese bereits im Speicher enthalten. Da eine Verzögerung nur bei wenigen Ansichten, wie der Navigation von der Upgradeauswahl in die Zusammenfassung auftritt, ist dieses Problem aufgrund der anderen positiven Bewertungen bei dem Kriterium der schnellen Rückmeldung zu vernachlässigen. Zur Lösung dieses Problems sollten die Daten vorgeladen werden und in der Wartezeit ein Ladebalken angezeigt werden. Eine andere Situation ist die Bewertung der Darstellung des aktuellen Konfigurationsstatus. Dies ist für die Anwendung essentiell, da der Benutzer zu jeder Zeit wissen muss, in welchem Status er sich befindet. Da auch hier die anderen Bewertungen positiv waren, ist keine deutliche Ursache des Problems auszumachen.

\subsection{Auswertung der Expertenergebnisse}
Bei den Experten sind die Fragen gezielt nach den Nicht-Funktionalen Anforderungen gestellt worden. Hierbei wurden die zehn Heuristiken von Nielsen als Grundlage verwendet und jeder dieser Punkte bewertet. Die Befragung hat die Stärken und Schwächen der Anwendung aufgezeigt. In Abbildung \ref ist das Ergebnis der Auswertung zu sehen. Für eine Flexibilität bei der Nutzung wurde der Expertenmodus (siehe \ref{expertDesign}) entworfen. Das Ziel des Modus wurde mit der positiven Bewertung eindeutig erreicht. Eine klare Umsetzung der Designvorgaben ist durch die gute Bewertung der Konsistenz und Standards sowie des ästhetischen und minimalistischen Designs gegeben. \par 
\begin{figure}[H]
\includegraphics{images/bewertung_expert_complete}
\caption{Auszug aus der Expertenbefragung}
\label{bewertungExpert}
\end{figure}
Die Heuristiken Hilfe und Dokumentation sowie Fehlerbehandlung sind weniger gut bewertet worden, da der Fokus der Nicht-Funktionalen Anforderungen auf anderen Kriterien lag. Diese Punkte sind außerdem weniger relevant, da ein Mitarbeiter des Herstellers bei der Anwendung der App dabei ist. Die Punkte Fehlertoleranz, Benutzerkontrolle und System stimmt mit der Wirklichkeit überein erhalten ebenfalls eine gute Bewertung, obwohl diese nicht im Vordergrund standen. Dies zeigt ein positives Gesamtbild der Anwendung. \par 
Auch bei den Experten ist die dauerhafte Sichtbarkeit des Systemstatus, der analog zum Konfigurationsstatus aus den Benutzertests zu sehen ist, weniger vorhanden. Der Punkt Erkennung statt Erinnerung ist mit dieser Bewertung verwandt und erhält deshalb eine schlechtere Bewertung.  An dieser Stelle fehlt ein Leitfaden in der Anwendung, so dass die folgenden Schritte deutlicher werden. 

\section{Konsequenzen aus der Evaluation}
Die Ergebnisse der Evaluation haben zwei Probleme der Anwendung aufgeworfen. Zum einen ist der Abschluss des Auswahlprozesses sowie der aktuelle Status der Konfiguration, bzw. des Systems nicht sichtbar. Beide Schwerpunkte beziehen sich auf die Zusammenfassung. Diese Seite ist als Orientierungshilfe für den Anwender konzipiert worden. Aus diesem Grund muss diese Ansicht den aktuellen Systemstatus und den Abschluss der Konfiguration beinhalten.  
\begin{figure}[H]
\includegraphics[width=\hsize]{images/impl/summary_impl}
\caption{Neue Zusammenfassungsansicht, nach dem Redesign}
\label{redesignSummary}
\end{figure}
Abbildung \ref{redesignSummary} zeigt den neuen Entwurf der Zusammenfassung. Die Idee ist der Mathematik entnommen, bei der Eingabe A + Eingabe B = Ergebnis C ergibt. Angewendet auf die Ansicht bedeutet dies, dass zuerst die Eingabedaten Flugzeug und Upgrades angezeigt werden und als Resultat die Konfigurationsgruppen gebildet werden. Weiterhin wird die Navigation zur Flugzeug- und Upgradeauswahl mit einer Kachel durchgeführt. Für eine bessere Übersicht erhalten die Kacheln die Anzahl der ausgewählten Elemente. Dies verhindert ein zu langes Scrollen bei größeren Selektionen. 
Der Abschluss der Konfiguration wird durch ein dauerhaftes Anzeigen der AppBar verbessert. Hierdurch sind die Buttons für einen Abschluss der Konfiguration dauerhaft sichtbar.\par 

Für eine zukünftige Entwicklung der App sind die weiteren Heuristiken interessant. Diese waren im Evaluationsbogen der Experten, jedoch nicht in der Zielsetzung der Anforderungen enthalten. Aufgrund der unterschiedlichen Bewertung der Experten in diesen Kriterien können hier weitere Verbesserungen durch die Umsetzung der Punkte erzielt werden. \par 
	\chapter{Fazit und Ausblick}\label{chapter_7}
In der vorliegenden Arbeit wurde ein Weg gezeigt, wie ein komplexes Produkt übersichtlich dargestellt werden kann. Nach einer Einführung in die Thematik und die Anwendungsdomäne wurde eine ausgiebige Analyse des Ist-Zustandes eines Konfigurationsprozesses durchgeführt. Die aufgetretenen Probleme sind in einem neuen Prozess behoben worden, der zusätzliche Anpassungen aufgrund der mobilen Zielumgebung enthält. Aus dem neuen Workflow sind die Anforderungen der Anwendung spezifiziert worden. Diese sind für eine Auswahl des passenden mobilen Anwendungstypen, sowie der entsprechenden Technologie verwendet worden. Nach der Auswahl der Plattform sind die einzelnen Prozessschritte für die mobile App konzipiert worden. Die folgende Implementierung wurde mit einem passenden Architektur-Pattern umgesetzt. Eine Evaluation am Ende hat die Erfüllung der Ziele bestätigt und konnte weitere Verbesserungsmöglichkeiten aufzeigen.

Die zugrundeliegenden Komplexität des Produktes wurde durch eine übersichtliche Aufbereitung des Inhalts in einer mobilen, touchgesteuerten Umgebung erreicht. Hierbei wurden besondere Bedienelemente der ausgewählten Technologie für eine einfache und schnelle Bedienung erfolgreich eingesetzt. Der Kunde kann die Anwendung damit genauso bedienen wie ein Mitarbeiter des Herstellers. 

Innerhalb der Arbeit ist eine Verbindung zwischen einem Produktkatalog und einer Konfigurationslösung zu einer gemeinsamen Komponente entstanden.  Durch die Software aus einer Hand können die Geschäftsprozesse für eine Individualisierung des Produktes schneller durchgeführt werden und der Kunde erhält ein schnelleres Feedback zu seiner Auswahl. Die App ist somit für eine Vereinfachung des Konfigurationsprozesses bereit und kann einen ökonomischen und komfortablen Weg der Konfiguration bieten, wodurch ein deutlicher Mehrwert beim Einsatz entsteht.  \par 

Der neue mobile Workflow wurde für ein konkretes Beispiel aus der Luftfahrt umgesetzt. Für die Zukunft ist eine Anwendung des neuen Prozesses auf weitere Geschäftsfelder interessant. Voraussetzung für die Umsetzung ist ein komplexes Produkt mit vielen Abhängigkeiten der einzelnen Bauteile, die für einen Kunden verständlich dargestellt werden müssen. Aus der Umsetzung auf einem weiteren Gebiet kann der modellierte Prozess weiter verbessert werden, damit die Effizienz bei der Durchführung einer Konfiguration für die Individualisierung eines Produktes weiter steigt.
	%\chapter{Fazit und Ausblick}\label{chapter_8}

	\begin{appendix}
		\chapter{Anhang}


\section{Quellcodeauszug}
Im der Abbildung \ref{aircraftFamilySelectionViewModelClass} ist das vollständige ViewModel für die Auswahl des Flugzeugprogramms zu sehen. Die Klasse wurde für ein Beispiel der Navigation in der Anwendung verwendet und ist im Text referenziert.

\begin{lstlisting}[caption=Vollständige SelectAircraftFamilyViewModel Klasse für die Flugzeugprogrammauswahl]
/// <summary>
    /// This View Model has the logic for aircraft family selection. Normally its the first view if
    /// a new configuration is started.
    /// </summary>
    public class SelectAircraftFamilyViewModel : GridHolderViewModel
    {
        private AircraftModel _model;
        private ICommand _familySelectedCommand;

        public SelectAircraftFamilyViewModel()
        {
            _model = new AircraftModel();
            InitializeDataSource();
        }

        private void InitializeDataSource()
        {

            DataGroupElements = new ObservableCollection<DataCommon>
                {new AircraftProgrammGroup(_model.GetAllAircraftProgramms())}; 
        }
        public ICommand SelectAircraftCommand
        {
            get { return _familySelectedCommand ?? (_familySelectedCommand = new RelayCommand<DataCommon>(SaveSelectionAndNavigateToSummaryPage)); }
            set
            {
                _familySelectedCommand = value;
                OnPropertyChanged();
            }
        }

        private void SaveSelectionAndNavigateToSummaryPage(DataCommon data)
        {
            var selectedProgramm = GetSelectedProgramm(data.UniqueId);
            _model.SelectAircraftProgramm(selectedProgramm);
            var classToNavigate = SimpleIoc.Default.GetInstance<ISummary>();
            var navigationService = SimpleIoc.Default.GetInstance<INavigationService>();
            navigationService.Navigate(classToNavigate.GetType());
        }

        private AircraftProgramm GetSelectedProgramm(string uniqueId)
        {
            return _model.GetAllAircraftProgramms().FirstOrDefault(programm => programm.UniqueId.Equals(uniqueId));
        }
    }
\end{lstlisting} 
\label{aircraftFamilySelectionViewModelClass}

 
\section{Evaluationsergebnisse} \label{anhangEva}
Im Folgenden werden die kompletten Ergebnisse der Evaluation dargestellt. Die Fragebögen für die Experten (siehe \ref{evaluationBogenExpert}) und der Benutzer (siehe \ref{evaluationSheetUser)


\includepdf[pages=1]{anhang/evaluationsbogenExperte.pdf}
\includepdf[pages=1-2]{anhang/evaluationsbogenUser.pdf}

	\end{appendix}
	
	% Anhang
	\clearpage
	\pagenumbering{roman}

	% Abbildungsverzeichnis
	\listoffigures
	\addcontentsline{toc}{chapter}{Abbildungsverzeichnis}
	
	\listoftables
	\addcontentsline{toc}{chapter}{Tabellenverzeichnis}
	
	
	% Literaturverzeichnis
	\clearpage
	\phantomsection
	\addcontentsline{toc}{chapter}{Literaturverzeichnis}
	\begin{thebibliography}{---}

 \bibitem[MASS]{bib:massCustomization}
           \textsc{Prof. Dr. Daniel Markgraf }
          \textbf{Wirtschaftslexikon}
          Mass Customization, http://wirtschaftslexikon.gabler.de/Definition/mass-customization.html, aufgerufen am 27.05.2013
          
 \bibitem[PUPPE]{bib:puppe}
 			 \textsc{Frank Puppe}
 			 \textbf{Springerverlag}
 			 Einführung in Expertensysteme, 1. Auflag, 1988
 			 
 \bibitem[KELLER]{bib:keller}
  			 \textsc{Hubert B. Keller}
  			 \textbf{DH Karlsruhe}
  			 Wissensbasierte Systeme - Einführung, Vorlesung SS 2007
  			 
   \bibitem[EXPERT]{bib:expert1}
             \textsc{Gerald Reif}
            \textbf{Deutsches Forschungszentrum für Knstliche Intelligenz}
            Expertensysteme, http://www.dfki.uni-kl.de/~aabecker/Mosbach/Experten/Reif-node8.html, aufgerufen am 18.07.2013
            
  \bibitem[RCP]{bib:eclipseRCP}
               \textsc{Ralf Ebert}
              Eclipse RCP, http://www.ralfebert.de/eclipse\_rcp/EclipseRCP.pdf, Version 1.1, 19.08.2011
  			
 			
\bibitem[ERGO]{bib:softwareErgonomie}
           \textsc{Christiane Rudlof}
          \textbf{Unfallkasse Post und Telekom}
          Handbuch Software-Ergonomie. Usebility Engineering., http://www.ukpt.de/pages/dateien/software-ergonomie.pdf, S.52, 2. Auflage, Tübingen 2006

 \bibitem[PLATTFORM]{bib:mobilePlattform}
           \textsc{Cloudsherpas}
          Native, Hybrid and Mobile Web Apps, http://www.cloudsherpas.com/services/custom-development/mobile-apps/native-hybrid-and-mobile-web-application-development/, aufgerufen am 23.07.2013

 \bibitem[WIN8-1]{bib:win81}
           \textsc{Bart Claeys, Qixing Zheng}
          \textbf{MSDN}
          Designfallstudie: vom iPad zur Windows Store-App, http://msdn.microsoft.com/de-de/library/windows/apps/hh868262, aufgerufen am 16.05.2013
          
  \bibitem[WIN8-2]{bib:win82}
            \textsc{Microsoft}
           \textbf{MSDN}
           Entwerfen großartiger Produktivitäts-Apps für Windows, http://msdn.microsoft.com/de-de/library/windows/apps/hh868273, aufgerufen am 16.05.2013
           
   \bibitem[WIN8-3]{bib:win83}
              \textsc{Microsoft}
             \textbf{MSDN}
             Shopping-Apps, http://msdn.microsoft.com/de-de/library/windows/apps/jj635241.aspx, aufgerufen am 16.05.2013
             
     \bibitem[WIN8-4]{bib:win84}
                \textsc{Microsoft}
               \textbf{MSDN}n am 16.05.2013
 Planen ihrer App, http://msdn.microsoft.com/de-de/library/windows/apps/hh465427, aufgerufe

 
\end{thebibliography}


	
	% Abkürzungsverzeichnis
	% vorher in Konsole folgendes aufrufen: 
	%	makeglossaries makeglossaries dokumentation.acn && makeglossaries dokumentation.glo
	\printglossary[type=\acronymtype]
	
	% Glossar
	\printglossary[style=altlist,title=Glossar]
	
	
\end{document}

