\thispagestyle{empty}

\renewcommand{\abstractname}{abstract}
\begin{abstract}
Die sogenannte Mass Customization ist ein Ansatz für die Herstellung eines individuellen Produktes. Voraussetzung für diese Art der Fertigung ist eine große Anzahl von individuellen Einzelbauteilen. Die Herausforderung besteht in der Vermittlung des komplexen Aufbaus eines solchen Produktes, so dass der Kunde eine individuelle Zusammenstellung durchführen kann. Gleichzeitig müssen komplexe Abhängigkeiten zwischen den Bauteilen beachtet werden, um ein Produkt technisch realisieren zu können. In diese Arbeit wird eine Lösung aufgezeigt, die einen Touchscreen auf einem mobilen Endgerät für eine übersichtliche Darstellung der Produktstruktur in einem Produktkatalog verwendet. Gleichzeitig findet durch den Einsatz eines Produktkonfigurators eine Überprüfung der aktuellen Zusammenstellung statt. Das Zusammenführen dieser beiden Komponenten führt dabei zu einer Vereinfachung des gesamten Konfigurationsprozesses und ermöglicht eine Beschleunigung der Prozesse. Der Einsatz des mobilen Endgeräts erzeugt ein besseres Verständnis für die Struktur des Produktes.


\end{abstract}

