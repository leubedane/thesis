\thispagestyle{empty}

\renewcommand{\abstractname}{abstract}
\begin{abstract}
Das Voranschreiten der sogenannten mass-customization bei Produkten erfordert immer komplexere Produkte, um ein hohes Maß an Individualität erreichen zu können. Für den Kunden im Mittelpunkt steht die Auswahl der einzelnen Komponenten des Produktes. Die Komplexität der Produkte soll für den Kunden nicht sichtbar sein. Es muss somit ein Weg gefunden werden, wie eine komplexe Produktlandschaft für den Kunden vereinfacht dargestellt werden kann. Im Rahmen dieser Bachelorthesis wird mit der Umsetzung eines Produktkataloges auf eine mobile Zielumgebung versucht dieses Ziel zu erreichen. Hierbei werden die wichtigsten Bedürfnisse des Kunden analysiert und darauf aufbauend eine Anwendung konzipiert. Durch die optimierte Darstellung der einzelnen Produkte, sowie eine Überprüfung der Zusammenstellung im Hintergrund wird ein Mehrwert für den Kunden erzielt. Die Sicherstellung der Zielerreichung wurde durch eine Evaluation der Lösung erreicht.

\end{abstract}


%\renewcommand{\abstractname}{Summary}
%\begin{abstract}
%An abstract is a brief summary of a research article, thesis, review,
%conference proceeding or any in-depth analysis of a particular subject
%or discipline, and is often used to help the reader quickly ascertain
%the paper's purpose. When used, an abstract always appears at the
%beginning of a manuscript, acting as the point-of-entry for any given
%scientific paper or patent application. Abstracting and indexing
%services for various academic disciplines are aimed at compiling a
%body of literature for that particular subject.

%The terms précis or synopsis are used in some publications to refer to
%the same thing that other publications might call an "abstract". In
%management reports, an executive summary usually contains more
%information (and often more sensitive information) than the abstract
%does.

%Quelle: \url{http://en.wikipedia.org/wiki/Abstract_(summary)}

%\end{abstract}
