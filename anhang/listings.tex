\chapter{Anhang}


\section{Klassendiagramm}
\begin{figure}
\begin{lstlisting}
/// <summary>
    /// This View Model has the logic for aircraft family selection. Normally its the first view if
    /// a new configuration is started.
    /// </summary>
    public class SelectAircraftFamilyViewModel : GridHolderViewModel
    {
        private AircraftModel _model;
        private ICommand _familySelectedCommand;

        public SelectAircraftFamilyViewModel()
        {
            _model = new AircraftModel();
            InitializeDataSource();
        }

        private void InitializeDataSource()
        {

            DataGroupElements = new ObservableCollection<DataCommon>
                {new AircraftProgrammGroup(_model.GetAllAircraftProgramms())}; 
        }
\end{lstlisting} 
\begin{lstlisting}
        public ICommand SelectAircraftCommand
        {
            get { return _familySelectedCommand ?? (_familySelectedCommand = new RelayCommand<DataCommon>(SaveSelectionAndNavigateToSummaryPage)); }
            set
            {
                _familySelectedCommand = value;
                OnPropertyChanged();
            }
        }

        private void SaveSelectionAndNavigateToSummaryPage(DataCommon data)
        {
            var selectedProgramm = GetSelectedProgramm(data.UniqueId);
            _model.SelectAircraftProgramm(selectedProgramm);
            var classToNavigate = SimpleIoc.Default.GetInstance<ISummary>();
            var navigationService = SimpleIoc.Default.GetInstance<INavigationService>();
            navigationService.Navigate(classToNavigate.GetType());
        }

        private AircraftProgramm GetSelectedProgramm(string uniqueId)
        {
            return _model.GetAllAircraftProgramms().FirstOrDefault(programm => programm.UniqueId.Equals(uniqueId));
        }
    }
\end{lstlisting} 
\label{aircraftFamilySelectionViewModel}
\end{figure}
 
\section{Evaluationsergebnisse} \label{anhangEva}